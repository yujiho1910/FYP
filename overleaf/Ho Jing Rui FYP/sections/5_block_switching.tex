\section{Block Switching in \texorpdfstring{$R(n)$}{R(n)}}
Similar to the previous section, we consider Seidel switching on $R(n)$, but instead of a single vertex we focus on switching on cliques of size $n$ instead.

The resulting adjacency matrix is as follows:
\begin{align*}
    A(R'(n)) = \begin{bmatrix}
        A(R_{k,n}) & C \\
        C^T & A(R_{n-k, n})
    \end{bmatrix}
\end{align*}

where $C = J_{k, n-k} \otimes (J-I)_n$.

\subsection{Switching Exactly Half the Vertices}
We first consider a specific case where $n$ is even, and we switch $k$ $n$-cliques, $k=n/2$. 
\subsubsection{Graph Construction and Symmetric Partitioning}
Since $k = n/2$, we rewrite our adjacency matrix as:
\begin{align*}
    A(\Gamma_{4}) = \begin{bmatrix}
        A(R_{k,2k}) & C \\
        C^T & A(R_{k, 2k})
    \end{bmatrix}
\end{align*}
where $C = J_{k} \otimes (J-I)_{2k}$.

\subsubsection{Coherent Algebra}

We claim that the following 6 matrices,

\[
\begin{aligned}
W_1 &= 
\begin{bmatrix}
I_{2k^2} & O_{2k^2} \\
O_{2k^2} & I_{2k^2}
\end{bmatrix},
& \quad
W_2 &= 
\begin{bmatrix}
(J_k - I) \otimes (J_{2k} - I) & O_{2k^2} \\
O_{2k^2} & (J_k - I) \otimes (J_{2k} - I)
\end{bmatrix}, \\[1em]
W_3 &= 
\begin{bmatrix}
O_{2k^2} & J_k \otimes I_{2k} \\
J_k \otimes I_{2k} & O_{2k^2}
\end{bmatrix},
& \quad
W_4 &= 
\begin{bmatrix}
I_k \otimes (J_{2k} - I) & O_{2k^2} \\
O_{2k^2} & I_k \otimes (J_{2k} - I)
\end{bmatrix}, \\[1em]
W_5 &= 
\begin{bmatrix}
(J_k - I) \otimes I_{2k} & O_{2k^2} \\
O_{2k^2} & (J_k - I) \otimes I_{2k}
\end{bmatrix},
& \quad
W_6 &= 
\begin{bmatrix}
O_{2k^2} & J_k \otimes (J_{2k} - I) \\
J_k \otimes (J_{2k} - I) & O_{2k^2}
\end{bmatrix}.
\end{aligned}
\]
form a basis for a coherent algebra that contains the adjacency matrix of$\Gamma_5$. We define this coherent algebra as:
\begin{equation*}
    \mathcal{A}(\Gamma_4) = \langle W_i: \text{ }i\in[6]\rangle.
\end{equation*}


\paragraph{Closure under Identity}
Since $\mathcal{W}_1 = I_{(2k)^2}$, the identity matrix exists in $\mathcal{A}(\Gamma_4)$.

\paragraph{Closure under Transpose}
It can be observed that $W_i, \quad i\in[6]$ are self-transpose, so $\mathcal{A}(\Gamma_4)$ is closed under transposition.

\paragraph{Closure under all-ones matrix}
If we sum all the matrices $\sum_{i=1}^{6}W_i$, we actually get $J_{(2k)^2}$, so the set $\mathcal{A}(\Gamma_4)$ contains the all-ones matrix.

\paragraph{Closure under matrix multiplication}

Here we have to show for each pair-wise multiplication, its product is still contained in $\mathcal{A}(\Gamma_4)$. 

In the appendix \ref{working:gamma-4}, we rigourously show that $\mathcal{A}(\Gamma_4)$ is closed under matrix multiplication.

Thus, $\mathcal{A}(\Gamma_4)$ is a coherent algebra. So we know the coherent closure $\mathcal{W}(\Gamma_4) \subseteq \mathcal{A}(\Gamma_4)$.

\subsubsection{Showing Minimal Coherent Algebra}

Recall that $\Gamma_4$ is the graph of $R(n)$ with $n/2$ $n$-cliques switched. Let $\mathcal{A}$ be an arbitrary coherent algebra containing the adjacency matrix $A(\Gamma_4)$, that is $A(\Gamma_4)=A\in\mathcal{A}$. Since any coherent algebra is closed under matrix multiplication, $A^2\in\mathcal{A}$. We show that $A^2$ can be expressed as a linear combination of certain classes of matrices grouped by their unique coefficients, which we show are classes in the coherent closure by Wielandt's Principle (Theorem \ref{def:wielandt-priniple}).

\begin{align*}
    A(\Gamma_4)^2 
    &= \begin{bmatrix}
        I_k\otimes (J_{2k}-I) + (J_k-I)\otimes I_{2k} & J_k\otimes (J_{2k}-I) \\
        J_k\otimes (J_{2k}-I) & I_k\otimes (J_{2k}-I) + (J_k-I)\otimes I_{2k}
    \end{bmatrix}^2 \\
    &= \begin{bmatrix}
        M_1 & M_2 \\
        M_2 & M_1
    \end{bmatrix}^2 \\
    &= \begin{bmatrix}
        M_1^2+M_2^2 & M_1M_2 + M_2M_1 \\
        M_1M_2 + M_2M_1 & M_1^2+M_2^2
    \end{bmatrix}
\end{align*}

We isolate the terms and solve it before substituting back into the matrix:
\begin{align*}
    M_1^2+M_2^2
    &= (I_k\otimes (J_{2k}-I) + (J_k-I)\otimes I_{2k})(I_k\otimes (J_{2k}-I) + (J_k-I)\otimes I_{2k})\\
    &\quad\quad\quad + (J_k\otimes (J_{2k}-I))(J_k\otimes (J_{2k}-I))\\
    &= (I_k\otimes (J_{2k}-I))(I_k\otimes (J_{2k}-I)) + (I_k\otimes (J_{2k}-I))(J_k-I)\otimes I_{2k}\\
    &\quad\quad\quad + ((J_k-I)\otimes I_{2k})(I_k\otimes (J_{2k}-I)) + ((J_k-I)\otimes I_{2k})((J_k-I)\otimes I_{2k})\\
    &\quad\quad\quad +(J^2_k\otimes (J_{2k}-I)^2)\\
    &= (I_k\otimes (J_{2k}-I)^2) + ((J_k-I)\otimes(J_{2k}-I))\\
    &\quad\quad\quad + ((J_k-I)\otimes(J_{2k}-I)) + ((J_k-I)^2 \otimes I_{2k})\\
    &\quad\quad\quad+(k(J_k-I)+k(I_k))\otimes ((2k-2)(J_{2k}-I) + (2k-1)(I_{2k}))\\
    &=(I_k\otimes ((2k-2)(J_{2k}-I) + (2k-1)I_{2k}))\\
    &\quad\quad\quad+2((J_k-I)\otimes(J_{2k}-I)) + ((k-2)(J_k-I) + (k-1)I_k)\otimes I_{2k}\\
    &\quad\quad\quad+k(2k-2)(J_k-I)\otimes(J_{2k}-I) + k(2k-2)I_k\otimes(J_{2k}-I) \\
    &\quad\quad\quad +k(2k-1)(J_k-I)\otimes I_{2k} + k(2k-1)I_k\otimes I_{2k}\\
    &= (2k^2+2k-2)I_k\otimes I_{2k}\\
    &\quad\quad\quad +(2k^2-2)(I_k\otimes(J_{2k}-I) + (J_k-I)\otimes I_{2k})\\
    &\quad\quad\quad+(2k^2-2k+2)(J_k-I)\otimes(J_{2k}-I)
\end{align*}

\begin{align*}
     M_1M_2 + M_2M_1
     &= (I_k\otimes (J_{2k}-I) + (J_k-I)\otimes I_{2k})(J_k\otimes (J_{2k}-I)) \\
     &\quad\quad\quad+(J_k\otimes (J_{2k}-I))(I_k\otimes (J_{2k}-I) + (J_k-I)\otimes I_{2k}) \\
     &= J_k\otimes (J_{2k}-I)^2 + (J_k-I)J_k\otimes (J_{2k}-I) \\
     &\quad\quad\quad+J_k\otimes (J_{2k}-I)^2 + J_k(J_k-I)\otimes (J_{2k}-I)\\
     &= 2(J_k\otimes (J_{2k}-I)^2 + (J_k-I)J_k\otimes (J_{2k}-I)) \\
     &=2(J_k\otimes [(2k-2)(J_{2k}-I)+(2k-1)I_{2k}] + (k-1)J_k\otimes (J_{2k}-I))\\
     &=2((2k-2+k-1)J_k\otimes (J_{2k}-I) + (2k-1)J_k\otimes I_{2k})\\
     &= (6k-6)J_k\otimes (J_{2k}-I) + (4k-2)J_k\otimes I_{2k}
\end{align*}

Substituting back into the matrix $A^2$,
\begin{align*}
    A(\Gamma_4)^2
    &= \begin{bmatrix}
        M_1^2+M_2^2 & M_1M_2 + M_2M_1 \\
        M_1M_2 + M_2M_1 & M_1^2+M_2^2
    \end{bmatrix}\\
    &= \begin{bmatrix}
        (2k^2+2k-2)I_k\otimes I_{2k}\\
     +(2k^2-2)(I_k\otimes(J_{2k}-I) + (J_k-I)\otimes I_{2k}) & (6k-6)J_k\otimes (J_{2k}-I)\\
    +(2k^2-2k+2)(J_k-I)\otimes(J_{2k}-I) & + (4k-2)J_k\otimes I_{2k} \\\\
        &(2k^2+2k-2)I_k\otimes I_{2k}\\
        (6k-6)J_k\otimes (J_{2k}-I) & +(2k^2-2)(I_k\otimes(J_{2k}-I) + (J_k-I)\otimes I_{2k}) \\
        +(4k-2)J_k\otimes I_{2k} & +(2k^2-2k+2)(J_k-I)\otimes(J_{2k}-I)
    \end{bmatrix} \\\\
    &= (2k^2+2k-2)\begin{bmatrix}
        I_k\otimes I_{2k} & O\\
        O & I_k\otimes I_{2k}
    \end{bmatrix} \\
    &\quad\quad+(2k^2-2k+2)\begin{bmatrix}
        (J_k-I)\otimes(J_{2k}-I) & O\\
        O & (J_k-I)\otimes(J_{2k}-I)
    \end{bmatrix} \\
    &\quad\quad(2k^2-2)\begin{bmatrix}
        (I_k\otimes(J_{2k}-I) + (J_k-I)\otimes I_{2k}) & O\\
        O & (I_k\otimes(J_{2k}-I) + (J_k-I)\otimes I_{2k})
    \end{bmatrix}\\
    &\quad\quad + (6k-6)\begin{bmatrix}
        O & J_k\otimes (J_{2k}-I)\\
        J_k\otimes (J_{2k}-I) & O
    \end{bmatrix} + (4k-2)\begin{bmatrix}
        O & J_k\otimes I_{2k}\\
        J_k\otimes I_{2k} & O
    \end{bmatrix}
\end{align*}

By Wielandt's Principle, the matrices below are classes of the coherent closure $\mathcal{W}(\Gamma_5)$:
\begin{align*}
    \operatorname{span}\{
    &\begin{bmatrix}
        I_k\otimes I_{2k} & O\\
        O & I_k\otimes I_{2k}
    \end{bmatrix}, 
    \begin{bmatrix}
        (J_k-I)\otimes(J_{2k}-I) & O\\
        O & (J_k-I)\otimes(J_{2k}-I)
    \end{bmatrix},\\
    &\begin{bmatrix}
        O & J_k\otimes (J_{2k}-I)\\
        J_k\otimes (J_{2k}-I) & O
    \end{bmatrix},
    \begin{bmatrix}
        O & J_k\otimes I_{2k}\\
        J_k\otimes I_{2k} & O
    \end{bmatrix}\\
    &\begin{bmatrix}
        (I_k\otimes(J_{2k}-I) + (J_k-I)\otimes I_{2k}) & O\\
        O & (I_k\otimes(J_{2k}-I) + (J_k-I)\otimes I_{2k})
    \end{bmatrix}
    \}\subseteq\mathcal{W}(\Gamma_4).
\end{align*}

We can now choose any 2 matrices from the set above and repeat the process to obtain more classes:

We choose to square $\begin{bmatrix}
        (I_k\otimes(J_{2k}-I) + (J_k-I)\otimes I_{2k}) & O\\
        O & (I_k\otimes(J_{2k}-I) + (J_k-I)\otimes I_{2k})
    \end{bmatrix}$,

\begin{align*}
    &\begin{bmatrix}
        (I_k\otimes(J_{2k}-I) + (J_k-I)\otimes I_{2k}) & O\\
        O & (I_k\otimes(J_{2k}-I) + (J_k-I)\otimes I_{2k})
    \end{bmatrix}^2.\\\\
    &=\begin{bmatrix}
        (I_k\otimes(J_{2k}-I) + (J_k-I)\otimes I_{2k})^2 & O\\
        O & (I_k\otimes(J_{2k}-I) + (J_k-I)\otimes I_{2k})^2
    \end{bmatrix} \\
    &=\begin{bmatrix}
        I_k\otimes (J_{2k}-I)^2+ (J_k-I)^2\otimes I_{2k} & O\\
        + 2(J_k-I)\otimes(J_{2k}-I) & \\
        O & I_k\otimes (J_{2k}-I)^2+ (J_k-I)^2\otimes I_{2k}\\
        &  + 2(J_k-I)\otimes(J_{2k}-I) 
    \end{bmatrix}\\
    &=\begin{bmatrix}
        I_k\otimes ((2k-2)(J_{2k}-I) + (2k-1)I_{2k}) & O\\
        + ((k-2)(J_k-I) + (k-1)I_k)\otimes I_{2k}& \\
        + 2(J_k-I)\otimes(J_{2k}-I) & \\
        O & I_k\otimes ((2k-2)(J_{2k}-I)\\
        & + ((k-2)(J_k-I) + (k-1)I_k)\otimes I_{2k}\\
        &  + 2(J_k-I)\otimes(J_{2k}-I) 
    \end{bmatrix}\\
    &=\begin{bmatrix}
        (3k-2)I_k\otimes I_{2k} & \\
        + (2k-2)I_k\otimes (J_{2k}-I)& \\
        + (k-2)(J_k-I)\otimes I_{2k} & O\\
        + 2(J_k-I)\otimes(J_{2k}-I)&\\
        & (3k-2)I_k\otimes I_{2k}\\
        & + (2k-2)I_k\otimes (J_{2k}-I)\\
        O&  + (k-2)(J_k-I)\otimes I_{2k} \\
        & + 2(J_k-I)\otimes(J_{2k}-I)
    \end{bmatrix}\\
    &= (3k-2)\begin{bmatrix}
        I_k\otimes I_{2k} & O\\
        O & I_k\otimes I_{2k}
    \end{bmatrix} + (2k-2)\begin{bmatrix}
        I_k\otimes (J_{2k}-I) & O\\
        O & I_k\otimes (J_{2k}-I)
    \end{bmatrix}\\
    +& (k-2)\begin{bmatrix}
        (J_k-I)\otimes I_{2k} & O\\
        O & (J_k-I)\otimes I_{2k}
    \end{bmatrix} + 2\begin{bmatrix}
        (J_k-I)\otimes(J_{2k}-I) & O\\
        O & (J_k-I)\otimes(J_{2k}-I)
    \end{bmatrix}
\end{align*}

Now we apply the Wielandt's Principle again to show the matrices below are classes of the coherent closure:

\begin{align*}
    \operatorname{span}\{
    &\begin{bmatrix}
        I_k\otimes I_{2k} & O\\
        O & I_k\otimes I_{2k}
    \end{bmatrix},
    \begin{bmatrix}
        I_k\otimes (J_{2k}-I) & O\\
        O & I_k\otimes (J_{2k}-I)
    \end{bmatrix}\\
    &\begin{bmatrix}
        (J_k-I)\otimes I_{2k} & O\\
        O & I_k\otimes (J_k-I)\otimes I_{2k}
    \end{bmatrix}\\
    &\begin{bmatrix}
        (J_k-I)\otimes(J_{2k}-I) & O\\
        O & (J_k-I)\otimes(J_{2k}-I)
    \end{bmatrix},\\
    &\begin{bmatrix}
        O & J_k\otimes (J_{2k}-I)\\
        J_k\otimes (J_{2k}-I) & O
    \end{bmatrix},
    \begin{bmatrix}
        O & J_k\otimes I_{2k}\\
        J_k\otimes I_{2k} & O
    \end{bmatrix}
    \}\subseteq\mathcal{W}(\Gamma_4)\\
    \Rightarrow &\mathcal{A}\subseteq\mathcal{W}(\Gamma_4).
\end{align*}

However, notice that $\mathcal{A}=\mathcal{A}(\Gamma_4)$, so we conclude that
\begin{align*}
    \mathcal{A}(\Gamma_4)=\mathcal{A}\subseteq\mathcal{W}(\Gamma_4)\\
    \Rightarrow\mathcal{A}(\Gamma_4)\subseteq\mathcal{W}(\Gamma_4).
\end{align*}

Since $\mathcal{A}(\Gamma_4)$ was proven to be a coherent algebra, $\mathcal{W}(\Gamma_4)\subseteq\mathcal{A}(\Gamma_4)$.
Therefore, $\mathcal{W}(\Gamma_4)=\mathcal{A}(\Gamma_4)$, and the coherent rank of $\Gamma_4$ is $|\mathcal{W}(\Gamma_4)| = |\langle W_i:\quad i\in[6]\rangle|=6$.

% To use the Wielandt Principle \ref{def:wielandt-priniple}, we let $A = A(\Gamma_4)\in\mathcal{A}$, so we know that $A^2\in\mathcal{A}$ as well.

% Detailed workings can be found in the Appendix \ref{working:lb-gamma-4}. For simplicity we will just state the relevant conclusions.


% \begin{align*}
%     A^2 
%     &= \begin{bmatrix}
%         I_k\otimes (J_{2k}-I) + (J_k-I)\otimes I_{2k} & J_k\otimes (J_{2k}-I) \\
%         J_k\otimes (J_{2k}-I) & I_k\otimes (J_{2k}-I) + (J_k-I)\otimes I_{2k}
%     \end{bmatrix}^2 \\\\
%     &= (2k^2+2k-2)\begin{bmatrix}
%         I_k\otimes I_{2k} & O\\
%         O & I_k\otimes I_{2k}
%     \end{bmatrix} \\
%     &\quad\quad+(2k^2-2k+2)\begin{bmatrix}
%         (J_k-I)\otimes(J_{2k}-I) & O\\
%         O & (J_k-I)\otimes(J_{2k}-I)
%     \end{bmatrix} \\
%     &\quad\quad(2k^2-2)\begin{bmatrix}
%         (I_k\otimes(J_{2k}-I) + (J_k-I)\otimes I_{2k}) & O\\
%         O & (I_k\otimes(J_{2k}-I) + (J_k-I)\otimes I_{2k})
%     \end{bmatrix}\\
%     &\quad\quad + (6k-6)\begin{bmatrix}
%         O & J_k\otimes (J_{2k}-I)\\
%         J_k\otimes (J_{2k}-I) & O
%     \end{bmatrix} + (4k-2)\begin{bmatrix}
%         O & J_k\otimes I_{2k}\\
%         J_k\otimes I_{2k} & O
%     \end{bmatrix}
% \end{align*}

% By Wielandt's Principle, 
% \begin{align*}
%     &\begin{bmatrix}
%         I_k\otimes I_{2k} & O\\
%         O & I_k\otimes I_{2k}
%     \end{bmatrix},\begin{bmatrix}
%         (J_k-I)\otimes(J_{2k}-I) & O\\
%         O & (J_k-I)\otimes(J_{2k}-I)
%     \end{bmatrix},\\
%     &\begin{bmatrix}
%         O & J_k\otimes (J_{2k}-I)\\
%         J_k\otimes (J_{2k}-I) & O
%     \end{bmatrix},\begin{bmatrix}
%         O & J_k\otimes I_{2k}\\
%         J_k\otimes I_{2k} & O
%     \end{bmatrix},\\
%     &\begin{bmatrix}
%         (I_k\otimes(J_{2k}-I) + (J_k-I)\otimes I_{2k}) & O\\
%         O & (I_k\otimes(J_{2k}-I) + (J_k-I)\otimes I_{2k})
%     \end{bmatrix} \in \mathcal{W}(\Gamma_4)
% \end{align*}

% Applying it one more time using the matrix 

% \[
% \begin{bmatrix}
%         (I_k\otimes(J_{2k}-I) + (J_k-I)\otimes I_{2k}) & O\\
%         O & (I_k\otimes(J_{2k}-I) + (J_k-I)\otimes I_{2k})
%     \end{bmatrix} \in \mathcal{W}(\Gamma_4)\subseteq\mathcal{A},
% \]
    
% \begin{align*}
%     &\begin{bmatrix}
%         (I_k\otimes(J_{2k}-I) + (J_k-I)\otimes I_{2k}) & O\\
%         O & (I_k\otimes(J_{2k}-I) + (J_k-I)\otimes I_{2k})
%     \end{bmatrix}^2\in\mathcal{A}(\Gamma_4)\\\\
%     =& (3k-2)\begin{bmatrix}
%         I_k\otimes I_{2k} & O\\
%         O & I_k\otimes I_{2k}
%     \end{bmatrix} + (2k-2)\begin{bmatrix}
%         I_k\otimes (J_{2k}-I) & O\\
%         O & I_k\otimes (J_{2k}-I)
%     \end{bmatrix}\\
%     +& (k-2)\begin{bmatrix}
%         (J_k-I)\otimes I_{2k} & O\\
%         O & (J_k-I)\otimes I_{2k}
%     \end{bmatrix} + 2\begin{bmatrix}
%         (J_k-I)\otimes(J_{2k}-I) & O\\
%         O & (J_k-I)\otimes(J_{2k}-I)
%     \end{bmatrix}
% \end{align*}

% By Wielandt's Principle, 
% \begin{align*}
%     &\begin{bmatrix}
%         I_k\otimes I_{2k} & O\\
%         O & I_k\otimes I_{2k}
%     \end{bmatrix},\begin{bmatrix}
%         I_k\otimes (J_{2k}-I) & O\\
%         O & I_k\otimes (J_{2k}-I)
%     \end{bmatrix}\\
%     &\begin{bmatrix}
%         (J_k-I)\otimes I_{2k} & O\\
%         O & (J_k-I)\otimes I_{2k}
%     \end{bmatrix}, \begin{bmatrix}
%         (J_k-I)\otimes(J_{2k}-I) & O\\
%         O & (J_k-I)\otimes(J_{2k}-I)
%     \end{bmatrix}\\
%     &\begin{bmatrix}
%         O & J_k\otimes (J_{2k}-I)\\
%         J_k\otimes (J_{2k}-I) & O
%     \end{bmatrix},\begin{bmatrix}
%         O & J_k\otimes I_{2k}\\
%         J_k\otimes I_{2k} & O
%     \end{bmatrix}\in\mathcal{W}(\Gamma_4)
% \end{align*}

% We have shown that there are at least 6 different classes in the coherent closure, so the coherent rank of this switched graph is $|\mathcal{W}(\Gamma_4)| = 6$. We can also conclude that 

% \[
% \mathcal{W}(\Gamma_4) = \langle W_1,W_2,W_3,W_4,W_5,W_6\rangle.
% \]

\newpage
\subsection{Switching \texorpdfstring{$k$}{k} Blocks of \texorpdfstring{$n$}{n} Vertices}
Now for a more general case, we choose to switch $k$ $n$-cliques from $R(n)$, $1<k<\lfloor n/2\rfloor$. 

\begin{remark}
    We restrict $k < \lfloor n/2 \rfloor$ as switching $k$ $n$-cliques is equivalent to switching $n-k$ $n$-cliques by symmetry of $R(n)$.
\end{remark}

\subsubsection{Graph Construction}
The resulting adjacency matrix is as follows:
\begin{align*}
     A(\Gamma_5) = \begin{bmatrix}
        A(R_{k,n}) & C \\
        C^T & A(R_{n-k, n})
    \end{bmatrix}
\end{align*}

where $C = J_{k, n-k} \otimes (J-I)_n$.

\subsubsection{Coherent Algebra}
We claim that the following 12 matrices,

\[
\begin{aligned}
W_1 &= 
\begin{bmatrix}
I_{kn} & O_{kn, (n-k)n} \\
O_{(n-k)n,kn} & O_{(n-k)n}
\end{bmatrix},
& \quad
W_2 &= 
\begin{bmatrix}
I_k \otimes (J_n - I) & O_{kn, (n-k)n} \\
O_{(n-k)n,kn} & O_{(n-k)n}
\end{bmatrix}, \\[1em]
W_3 &= 
\begin{bmatrix}
(J_k - I) \otimes I_n & O_{kn, (n-k)n} \\
O_{(n-k)n,kn} & O_{(n-k)n}
\end{bmatrix},
& \quad
W_4 &= 
\begin{bmatrix}
(J_k - I) \otimes (J_n - I) & O_{kn, (n-k)n} \\
O_{(n-k)n,kn} & O_{(n-k)n}
\end{bmatrix}, \\[1em]
W_5 &= 
\begin{bmatrix}
O_{kn} & O_{kn, (n-k)n} \\
O_{(n-k)n,kn} & I_{(n - k)n}
\end{bmatrix},
& \quad
W_6 &= 
\begin{bmatrix}
O_{kn} & O_{kn, (n-k)n} \\
O_{(n-k)n,kn} & I_{n - k} \otimes (J_n - I)
\end{bmatrix}, \\[1em]
W_7 &= 
\begin{bmatrix}
O_{kn} & O_{kn, (n-k)n} \\
O_{(n-k)n,kn} & (J_{n - k} - I) \otimes I_n
\end{bmatrix},
& \quad
W_8 &= 
\begin{bmatrix}
O_{kn} & O_{kn, (n-k)n} \\
O_{(n-k)n,kn} & (J_{n - k} - I) \otimes (J_n - I)
\end{bmatrix}, \\[1em]
W_9 &= 
\begin{bmatrix}
O_{kn} & J_{k,n-k} \otimes I_n \\
O_{(n-k)n,kn} & O_{(n-k)n}
\end{bmatrix},
& \quad
W_{10} &=
\begin{bmatrix}
O_{kn} & J_{k,n-k} \otimes (J_n - I) \\
O_{(n-k)n,kn} & O_{(n-k)n}
\end{bmatrix}, \\[1em]
W_{11} &=
\begin{bmatrix}
O_{kn} & O_{kn, (n-k)n} \\
J_{n-k, k}\otimes I_n & O_{(n-k)n}
\end{bmatrix}, 
& \quad
W_{12} &=
\begin{bmatrix}
O_{kn} & O_{kn, (n-k)n} \\
J_{n-k, k}\otimes (J_n-I) & O_{(n-k)n}
\end{bmatrix}
\end{aligned}
\]
form a basis for a coherent algebra that contains the adjacency matrix of$\Gamma_5$. We define this coherent algebra as:
\begin{equation*}
    \mathcal{A}(\Gamma_5) = \langle W_i: \text{ }i\in[12]\rangle.
\end{equation*}

\paragraph{Closure under Identity}
Since $\mathcal{W}_1 + \mathcal{W}_5 = I_{n^2}$, the identity matrix exists in $\mathcal{A}(\Gamma_5)$.

\paragraph{Closure under Transpose}
It can be observed that $W_i,\quad i\in[8]$ are self-transpose, and we verify that
\begin{align*}
    W_9^T = W_{11} \quad\text{and}\quad W_{10}^T = W_{12}
\end{align*}
so $\mathcal{A}(\Gamma_5)$ is closed under transposition.

\paragraph{Closure under all-ones matrix}
If we sum all the matrices $\sum_{i=1}^{12}W_i$, we actually get $J_{n^2}$, so the set $\mathcal{A}(\Gamma_5)$ contains the all-ones matrix.

\paragraph{Closure under matrix multiplication}

Here we have to show for each pair-wise multiplication, its product is still contained in $\mathcal{A}(\Gamma_5)$. 

In the appendix \ref{working:gamma-5}, we rigourously show that $\mathcal{A}(\Gamma_5)$ is closed under matrix multiplication.

Thus, $\mathcal{A}(\Gamma_5)$ is a coherent algebra. So we know the coherent closure $\mathcal{W}(\Gamma_5)\subseteq\mathcal{A}(\Gamma_5)$.

\subsubsection{Showing Minimal Coherent Algebra}

Recall that $\Gamma_5$ is the graph of $R(n)$ with $k$ $n$-cliques switched, $1<k<n/2$. Let $\mathcal{A}$ be an arbitrary coherent algebra containing the adjacency matrix $A(\Gamma_5)$, that is $A(\Gamma_5)=A\in\mathcal{A}$. Since any coherent algebra is closed under matrix multiplication, $A^2\in\mathcal{A}$. We show that $A^2$ can be expressed as a linear combination of certain classes of matrices grouped by their unique coefficients, which we show are classes in the coherent closure by Wielandt's Principle (Theorem \ref{def:wielandt-priniple}).

\begin{align*}
    A(\Gamma_5)^2
    &= \begin{bmatrix}
        I_k \otimes (J_n-I) + (J_k-I) \otimes I_n & J_{k,n-k}\otimes (J_n-I) \\
        J_{n-k,k}\otimes (J_n-I) & I_{n-k} \otimes (J_n-I) + (J_{n-k}-I)\otimes I_n
    \end{bmatrix}^2\\
    &= \begin{bmatrix}
        M_1 & M_2\\
        M_3 & M_4
    \end{bmatrix}^2\\
    &= \begin{bmatrix}
        M_1^2 + M_2M_3 & M_1M_2 + M_2M_4\\
        M_3M_1 + M_4M_3 & M_3M_2 + M_4^2
    \end{bmatrix}
\end{align*}

We isolate the terms and solve it before substituting back into the matrix:

\begin{align*}
    &M_1^2 + M_2M_3\\ 
    &= (I_k \otimes (J_n-I) + (J_k-I) \otimes I_n)^2 + (J_{k,n-k}\otimes (J_n-I))(J_{n-k,k}\otimes (J_n-I))\\
    &= (I_k \otimes (J_n-I)^2) + ((J_k-I)^2 \otimes I_n) + 2((J_k-I) \otimes(J_n-I)) + (J_{k,n-k}J_{n-k,k}\otimes (J_n-I)^2)\\
    &= (n-2)I_k\otimes(J_n-I) + (n-1)I_k\otimes I_n + (k-2)(J_k-I)\otimes I_n + (k-1)I_k\otimes I_n\\
    &\quad\quad+2((J_k-I) \otimes(J_n-I)) + (n-k)(n-2)(J_k-I) \otimes (J_n-I) + (n-k)(n-1)(J_k-I) \otimes I_n\\
    &\quad\quad\quad +(n-k)(n-2)I_k\otimes(J_n-I) + (n-k)(n-1)I_k\otimes I_n\\
    &= (n^2-kn+2k-2)I_k\otimes I_n \\
    &\quad\quad +(n^2-kn-n+2k-2)(I_k\otimes(J_n-I) + (J_k-I) \otimes I_n)\\
    &\quad\quad +(n^2-kn-2n+2k+2)((J_k-I) \otimes(J_n-I)).
\end{align*}

\begin{align*}
    &M_1M_2 + M_2M_4 \\
    &= (I_k \otimes (J_n-I) + (J_k-I) \otimes I_n)(J_{k,n-k}\otimes (J_n-I))\\
    &\quad\quad + (J_{k,n-k}\otimes (J_n-I))(I_{n-k} \otimes (J_n-I) + (J_{n-k}-I)\otimes I_n)\\
    &= J_{k,n-k}\otimes (J_n-I)^2 + (J_k-I)J_{k,n-k}\otimes (J_n-I)\\
    &\quad\quad + J_{k,n-k}\otimes (J_n-I)^2 + J_{k,n-k}(J_{n-k}-I)\otimes (J_n-I)\\
    &= 2J_{k,n-k}\otimes (J_n-I)^2 + (k-1 +n-k-1)J_{k,n-k}\otimes (J_n-I)\\
    &= 2(n-2)J_{k,n-k}\otimes (J_n-I) + 2(n-1)J_{k,n-k}\otimes I_n\\
    &\quad\quad + (n-2)J_{k,n-k}\otimes (J_n-I)\\
    &= (2n-2)J_{k,n-k}\otimes I_n + (3n-6)J_{k,n-k}\otimes (J_n-I).
\end{align*}

\begin{align*}
    &M_3M_1 + M_4M_3 \\
    &= (J_{n-k,k}\otimes (J_n-I))(I_k \otimes (J_n-I) + (J_k-I) \otimes I_n)\\
    &\quad\quad + (I_{n-k} \otimes (J_n-I) + (J_{n-k}-I)\otimes I_n)(J_{n-k,k}\otimes (J_n-I))\\
    &= J_{n-k,k}\otimes (J_n-I)^2 + J_{n-k,k}(J_k-I)\otimes (J_n-I) \\
    &\quad\quad + J_{n-k,k}\otimes (J_n-I)^2 + (J_{n-k}-I)J_{n-k,k}\otimes (J_n-I)\\
    &= 2J_{n-k,k}\otimes (J_n-I)^2 + (k-1 +n-k-1)J_{n-k,k}\otimes (J_n-I)\\
    &= 2(n-2)J_{n-k,k}\otimes (J_n-I) + 2(n-1)J_{n-k,k}\otimes I_n\\
    &\quad\quad + (n-2)J_{n-k,k}\otimes (J_n-I)\\
    &= (2n-2)J_{n-k,k}\otimes I_n + (3n-6)J_{n-k,k}\otimes (J_n-I).
    % &= ((M_3M_1 + M_4M_3)^T)^T\\
    % &= (M_1^TM_3^T + M_3^TM_4^T)^T\\
    % &= (M_1M_2+M_2M_4)^T\quad\text{(notice that }M_2 = M_3^T)\\
    % &= (
\end{align*}

\begin{align*}
    &M_3M_2 + M_4^2\\
    &= (J_{n-k,k}\otimes (J_n-I))(J_{k,n-k}\otimes (J_n-I)) + (I_{n-k} \otimes (J_n-I) + (J_{n-k}-I)\otimes I_n)^2\\
    &= J_{n-k,k}J_{k,n-k}\otimes (J_n-I)^2 + I_{n-k} \otimes (J_n-I)^2 + (J_{n-k}-I)^2\otimes I_n \\
    &\quad\quad + 2(J_{n-k}-I)\otimes(J_n-I)\\
    &= k(n-2)(J_{n-k}-I)\otimes(J_n-I) +  k(n-1)(J_{n-k}-I)\otimes I_n + k(n-2)I_k\otimes (J_n-I)\\
    &\quad\quad +k(n-1)I_k\otimes I_n + (n-2)I_{n-k}\otimes(J_n-I) + (n-1)I_{n-k}\otimes I_n\\
    &\quad\quad+ (n-k-2)(J_{n-k}-I)\otimes I_n + (n-k-1)I_{n-k}\otimes I_n + 2(J_{n-k}-I)\otimes(J_n-I)\\
    &= (kn-2k+2n-2)I_{n-k}\otimes I_n\\
    &\quad\quad+(kn-2k+n-2)(I_{n-k}\otimes(J_n-I)+(J_{n-k}-I)\otimes I_n)\\
    &\quad\quad+(kn-2k+2)(J_{n-k}-I)\otimes(J_n-I).
\end{align*}

Substituting back into the matrix $A^2$, 

\begin{align*}
    &A(\Gamma_5) ^2\\
    % &= \begin{bmatrix}
    %     (n^2-kn+2k-2)I_k\otimes I_n &\\
    % +(n^2-kn-n+2k-2)(I_k\otimes(J_n-I) + (J_k-I) \otimes I_n)& (2n-2)J_{k,n-k}\otimes I_n\\
    % +(n^2-kn-2n+2k+2)((J_k-I) \otimes(J_n-I))& + (3n-6)J_{k,n-k}\otimes (J_n-I) \\\\
    % &(kn-2k+2n-2)I_{n-k}\otimes I_n\\
    % (2n-2)J_{n-k,k}\otimes I_n  &+(kn-2k+n-2)(I_{n-k}\otimes(J_n-I)+(J_{n-k}-I)\otimes I_n)\\
    % + (3n-6)J_{n-k,k}\otimes (J_n-I)\otimes (J_n-I)&+ (kn-2k+2)(J_{n-k}-I)\otimes(J_n-I)
    % \end{bmatrix}\\
    &= (n^2-kn+2k-2)\begin{bmatrix}
        I_{kn} & O_{kn, (n-k)n} \\
        O_{(n-k)n,kn} & O_{(n-k)n}
    \end{bmatrix}\\
    &\quad+(n^2-kn-2n+2k+2)\begin{bmatrix}
        (J_k - I) \otimes (J_n - I) & O_{kn, (n-k)n} \\
        O_{(n-k)n,kn} & O_{(n-k)n}
    \end{bmatrix}\\
    &\quad+(n^2-kn-n+2k-2)\begin{bmatrix}
        I_k\otimes(J_n-I) + (J_k-I) \otimes I_n & O_{kn, (n-k)n} \\
        O_{(n-k)n,kn} & O_{(n-k)n}
    \end{bmatrix}\\
    &\quad+(2n-2)\begin{bmatrix}
        O_{kn} & J_{k,n-k} \otimes I_n \\
        J_{n-k,k}\otimes I_n & O_{(n-k)n}
    \end{bmatrix} + (3n-6)\begin{bmatrix}
        O_{kn} & J_{k,n-k} \otimes (J_n-I) \\
        J_{n-k,k}\otimes (J_n-I) & O_{(n-k)n}
    \end{bmatrix} \\
    &\quad +(kn-2k+2n-2)\begin{bmatrix}
        O_{kn} & O_{kn, (n-k)n} \\
        O_{(n-k)n,kn} & I_{(n - k)n}
    \end{bmatrix} + (kn-2k+2)\begin{bmatrix}
        O_{kn} & O_{kn, (n-k)n} \\
        O_{(n-k)n,kn} & (J_{n - k} - I) \otimes (J_n - I)
    \end{bmatrix}\\
    &\quad+(kn-2k+n-2)\begin{bmatrix}
        O_{kn} & O_{kn, (n-k)n} \\
        O_{(n-k)n,kn} & I_{n-k}\otimes(J_n-I)+(J_{n-k}-I)\otimes I_n
    \end{bmatrix}.
\end{align*}

By Wielandt's Principle, the matrices below are classes of the coherent closure $\mathcal{W}(\Gamma_5)$:

\begin{align*}
    \operatorname{span}\{
    &\begin{bmatrix}
        I_{kn} & O_{kn, (n-k)n} \\
        O_{(n-k)n,kn} & O_{(n-k)n}
    \end{bmatrix},
    \begin{bmatrix}
        (J_k - I) \otimes (J_n - I) & O_{kn, (n-k)n} \\
        O_{(n-k)n,kn} & O_{(n-k)n}
    \end{bmatrix},\\
    &\begin{bmatrix}
        I_k\otimes(J_n-I) + (J_k-I) \otimes I_n & O_{kn, (n-k)n} \\
        O_{(n-k)n,kn} & O_{(n-k)n}
    \end{bmatrix},
    \begin{bmatrix}
        O_{kn} & J_{k,n-k} \otimes I_n \\
        J_{n-k,k}\otimes I_n & O_{(n-k)n}
    \end{bmatrix},\\
    &\begin{bmatrix}
        O_{kn} & J_{k,n-k} \otimes (J_n-I) \\
        J_{n-k,k}\otimes (J_n-I) & O_{(n-k)n}
    \end{bmatrix},
    \begin{bmatrix}
        O_{kn} & O_{kn, (n-k)n} \\
        O_{(n-k)n,kn} & I_{(n - k)n}
    \end{bmatrix},\\
    &\begin{bmatrix}
        O_{kn} & O_{kn, (n-k)n} \\
        O_{(n-k)n,kn} & (J_{n - k} - I) \otimes (J_n - I)
    \end{bmatrix},
    \begin{bmatrix}
        O_{kn} & O_{kn, (n-k)n} \\
        O_{(n-k)n,kn} & I_{n-k}\otimes(J_n-I)+(J_{n-k}-I)\otimes I_n
    \end{bmatrix}
    \}\subseteq\mathcal{W}(\Gamma_5).
\end{align*}

We can now choose any 2 matrices from the set above and repeat the process to obtain more classes:

We choose to square $\begin{bmatrix}
        I_k\otimes(J_n-I) + (J_k-I) \otimes I_n & O_{kn, (n-k)n} \\
        O_{(n-k)n,kn} & O_{(n-k)n}
    \end{bmatrix}$,
\begin{align*}
    &\begin{bmatrix}
        I_k\otimes(J_n-I) + (J_k-I) \otimes I_n & O_{kn, (n-k)n} \\
        O_{(n-k)n,kn} & O_{(n-k)n}
    \end{bmatrix}^2\\
    &= \begin{bmatrix}
        I_k\otimes(J_n-I)^2 + (J_k-I)^2 \otimes I_n & \\
        +2(J_k-I)\otimes(J_n-I)&O_{kn, (n-k)n} \\
        O_{(n-k)n,kn} & O_{(n-k)n}
    \end{bmatrix}\\
    &= \begin{bmatrix}
        (n-2)I_k\otimes(J_n-I) + (n-1)I_k\otimes I_n + (k-2)(J_k-I) \otimes I_n  & \\
        + (k-1)I_k\otimes I_n +2(J_k-I)\otimes(J_n-I)&O_{kn, (n-k)n} \\
        O_{(n-k)n,kn} & O_{(n-k)n}
    \end{bmatrix}\\
    &= (n-k-2)\begin{bmatrix}
        I_{kn} & O_{kn, (n-k)n} \\
        O_{(n-k)n,kn} & O_{(n-k)n}
    \end{bmatrix} + (n-2)\begin{bmatrix}
        I_k\otimes(J_n-I) & O_{kn, (n-k)n} \\
        O_{(n-k)n,kn} & O_{(n-k)n}
    \end{bmatrix}\\
    &\quad + (k-2)\begin{bmatrix}
        (J_k-I) \otimes I_n & O_{kn, (n-k)n} \\
        O_{(n-k)n,kn} & O_{(n-k)n}
    \end{bmatrix} + 2\begin{bmatrix}
        (J_k-I)\otimes(J_n-I) & O_{kn, (n-k)n} \\
        O_{(n-k)n,kn} & O_{(n-k)n}
    \end{bmatrix}.
\end{align*}

We choose to square $\begin{bmatrix},
        O_{kn} & O_{kn, (n-k)n} \\
        O_{(n-k)n,kn} & I_{n-k}\otimes(J_n-I)+(J_{n-k}-I)\otimes I_n
    \end{bmatrix}$
\begin{align*}
    &\begin{bmatrix}
        O_{kn} & O_{kn, (n-k)n} \\
        O_{(n-k)n,kn} & I_{n-k}\otimes(J_n-I)+(J_{n-k}-I)\otimes I_n
    \end{bmatrix}^2\\
    &= \begin{bmatrix}
        O_{kn} & O_{kn, (n-k)n} \\
        O_{(n-k)n,kn} & I_{n-k}\otimes(J_n-I)^2 + (J_{n-k}-I)^2 \otimes I_n \\
        &+2(J_{n-k}-I)\otimes(J_n-I)
    \end{bmatrix}\\
    &= \begin{bmatrix}
        O_{kn} & O_{kn, (n-k)n} \\
        O_{(n-k)n,kn} & (n-2)I_{n-k}\otimes(J_n-I) + (n-1)I_{n-k}\otimes I_n + (n-k-2)(J_{n-k}-I) \otimes I_n \\
        &+(n-k-1)I_{n-k}\otimes I_n +2(J_{n-k}-I)\otimes(J_n-I)
    \end{bmatrix}\\
    &= (2n-k-2)\begin{bmatrix}
        O_{kn} & O_{kn, (n-k)n} \\
        O_{(n-k)n,kn} & I_{(n-k)n}
    \end{bmatrix} + (n-2)\begin{bmatrix}
         O_{kn} & O_{kn, (n-k)n} \\
        O_{(n-k)n,kn} & I_{n-k}\otimes(J_n-I)
    \end{bmatrix}\\
    &\quad + (n-k-2)\begin{bmatrix}
         O_{kn} & O_{kn, (n-k)n} \\
        O_{(n-k)n,kn} &(J_{n-k}-I) \otimes I_n
    \end{bmatrix} + 2\begin{bmatrix}
         O_{kn} & O_{kn, (n-k)n} \\
        O_{(n-k)n,kn} & (J_{n-k}-I)\otimes(J_n-I)
    \end{bmatrix}.
\end{align*}

We choose to multiply $\begin{bmatrix}
        O_{kn} & J_{k,n-k} \otimes I_n \\
        J_{n-k,k}\otimes I_n & O_{(n-k)n}
    \end{bmatrix}$ with $\begin{bmatrix}
        I_{kn} & O_{kn, (n-k)n} \\
        O_{(n-k)n,kn} & O_{(n-k)n}
    \end{bmatrix}$ and \\$\begin{bmatrix}
        O_{kn} & O_{kn, (n-k)n} \\
        O_{(n-k)n,kn} & I_{(n - k)n}
    \end{bmatrix}$ respectively,

\begin{align*}
    &\begin{bmatrix}
        O_{kn} & J_{k,n-k} \otimes I_n \\
        J_{n-k,k}\otimes I_n & O_{(n-k)n}
    \end{bmatrix}\begin{bmatrix}
        I_{kn} & O_{kn, (n-k)n} \\
        O_{(n-k)n,kn} & O_{(n-k)n}
    \end{bmatrix}\\
    &=\begin{bmatrix}
        O_{kn} & O_{kn, (n-k)n} \\
        J_{n-k,k}\otimes I_n & O_{(n-k)n}
    \end{bmatrix}.\\\\
    &\begin{bmatrix}
        O_{kn} & J_{k,n-k} \otimes I_n \\
        J_{n-k,k}\otimes I_n & O_{(n-k)n}
    \end{bmatrix}\begin{bmatrix}
        O_{kn} & O_{kn, (n-k)n} \\
        O_{(n-k)n,kn} & I_{(n - k)n}
    \end{bmatrix}\\
    &= \begin{bmatrix}
        O_{kn} & J_{k,n-k} \otimes I_n \\
        O_{(n-k)n,kn} & O_{(n-k)n}
    \end{bmatrix}.
\end{align*}

Similarly, we choose to multiply $\begin{bmatrix}
        O_{kn} & J_{k,n-k} \otimes (J_n-I) \\
        J_{n-k,k}\otimes (J_n-I) & O_{(n-k)n}
    \end{bmatrix}$ with $\begin{bmatrix}
        I_{kn} & O_{kn, (n-k)n} \\
        O_{(n-k)n,kn} & O_{(n-k)n}
    \end{bmatrix}$ and \\$\begin{bmatrix}
        O_{kn} & O_{kn, (n-k)n} \\
        O_{(n-k)n,kn} & I_{(n - k)n}
    \end{bmatrix}$ respectively,

\begin{align*}
    &\begin{bmatrix}
        O_{kn} & J_{k,n-k} \otimes (J_n-I) \\
        J_{n-k,k}\otimes (J_n-I) & O_{(n-k)n}
    \end{bmatrix}\begin{bmatrix}
        I_{kn} & O_{kn, (n-k)n} \\
        O_{(n-k)n,kn} & O_{(n-k)n}
    \end{bmatrix}\\
    &=\begin{bmatrix}
        O_{kn} & O_{kn, (n-k)n} \\
        J_{n-k,k}\otimes (J_n-I) & O_{(n-k)n}
    \end{bmatrix}.\\\\
    &\begin{bmatrix}
        O_{kn} & J_{k,n-k} \otimes (J_n-I) \\
        J_{n-k,k}\otimes (J_n-I) & O_{(n-k)n}
    \end{bmatrix}\begin{bmatrix}
        O_{kn} & O_{kn, (n-k)n} \\
        O_{(n-k)n,kn} & I_{(n - k)n}
    \end{bmatrix}\\
    &= \begin{bmatrix}
        O_{kn} & J_{k,n-k} \otimes (J_n-I) \\
        O_{(n-k)n,kn} & O_{(n-k)n}
    \end{bmatrix}.
\end{align*}

Now, we apply the Wielandt's Principle again to show the matrices below are classes of the coherent closure:

\begin{align*}
    \operatorname{span}\{
    &\begin{bmatrix}
        I_{kn} & O_{kn, (n-k)n} \\
        O_{(n-k)n,kn} & O_{(n-k)n}
    \end{bmatrix},
    \begin{bmatrix}
        I_k\otimes(J_n-I) & O_{kn, (n-k)n} \\
        O_{(n-k)n,kn} & O_{(n-k)n}
    \end{bmatrix},\\
    &\begin{bmatrix}
        (J_k-I)\otimes I_n & O_{kn, (n-k)n} \\
        O_{(n-k)n,kn} & O_{(n-k)n}
    \end{bmatrix},
    \begin{bmatrix}
        (J_k-I)\otimes (J_n-I) & O_{kn, (n-k)n} \\
        O_{(n-k)n,kn} & O_{(n-k)n}
    \end{bmatrix},\\
    &\begin{bmatrix}
        O_{kn} & O_{kn, (n-k)n} \\
        O_{(n-k)n,kn} & I_{(n-k)n}
    \end{bmatrix},
    \begin{bmatrix}
         O_{kn} & O_{kn, (n-k)n} \\
        O_{(n-k)n,kn} & I_{n-k}\otimes(J_n-I)
    \end{bmatrix},\\
    &\begin{bmatrix}
         O_{kn} & O_{kn, (n-k)n} \\
        O_{(n-k)n,kn} &(J_{n-k}-I) \otimes I_n
    \end{bmatrix},
    \begin{bmatrix}
         O_{kn} & O_{kn, (n-k)n} \\
        O_{(n-k)n,kn} & (J_{n-k}-I)\otimes(J_n-I)
    \end{bmatrix},\\
    &\begin{bmatrix}
        O_{kn} & J_{k,n-k} \otimes I_n \\
        O_{(n-k)n,kn} & O_{(n-k)n}
    \end{bmatrix},
    \begin{bmatrix}
        O_{kn} & O_{kn, (n-k)n} \\
        J_{n-k,k}\otimes I_n & O_{(n-k)n}
    \end{bmatrix},\\
    &\begin{bmatrix}
        O_{kn} & J_{k,n-k} \otimes (J_n-I) \\
        O_{(n-k)n,kn} & O_{(n-k)n}
    \end{bmatrix},
    \begin{bmatrix}
        O_{kn} & O_{kn, (n-k)n} \\
        J_{n-k,k}\otimes (J_n-I) & O_{(n-k)n}
    \end{bmatrix}
    \}\subseteq\mathcal{W}(\Gamma_5)\\
    \Rightarrow &\mathcal{A}\subseteq\mathcal{W}(\Gamma_5).
\end{align*}

However, notice that $\mathcal{A}=\mathcal{A}(\Gamma_5)$, so we conclude that
\begin{align*}
    \mathcal{A}(\Gamma_5)=\mathcal{A}\subseteq\mathcal{W}(\Gamma_5)\\
    \Rightarrow\mathcal{A}(\Gamma_5)\subseteq\mathcal{W}(\Gamma_5).
\end{align*}

Since $\mathcal{A}(\Gamma_5)$ was proven to be a coherent algebra, $\mathcal{W}(\Gamma_5)\subseteq\mathcal{A}(\Gamma_5)$.
Therefore, $\mathcal{W}(\Gamma_5)=\mathcal{A}(\Gamma_5)$, and the coherent rank of $\Gamma_5$ is $|\mathcal{W}(\Gamma_5)| = |\langle W_i:\quad i\in[12]\rangle|=12$.


% To use the Wielandt Principle \ref{def:wielandt-priniple}, we let $A=A(\Gamma_5)\in\mathcal{A}$, so we know that $A^2\in\mathcal{A}$ we well.

% Detailed workings can be found in the appendix \ref{working:lb-gamma-5}. For simplicity we will just state the relevant conclusions.

% We have shown that there are at least 12 different classes in the coherent closure, so the coherent rank of this switched graph is $|\mathcal{W}(\Gamma_5)| = 12$. We can also conclude that

% \begin{equation*}
%     \mathcal{W}(\Gamma_5) = \langle W_1, W_2, W_3, W_4, W_5, W_6, W_7, W_8, W_9, W_{10}, W_{11}, W_{12} \rangle.
% \end{equation*}
