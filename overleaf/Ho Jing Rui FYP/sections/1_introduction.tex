\section{Introduction}

In this paper, we investigate the coherent closure of graphs derived by performing graph operations on strongly regular graphs. We first discuss the motivation and why we choose to investigate such properties.

\subsection{Historical Motivation}

In 1782, the mathematician Leonhard Euler posed a question: There are 6 army regiments, each with 6 officers of varying ranks. Is there a way to arrange the 36 officers in a 6-by-6 square such that no row nor column have any repeated army regiments or ranks? \cite{euler-36-officers-problem}. This question, now known as Euler's 36 Officer problem, was deemed impossible at the time, but inspired the study of what we now know as Mutually Orthogonal Latin Squares. By definition, a Latin Square is a $n$-by-$n$ array filled with $n$ different symbols such that no rows or columns have a duplicate symbol. A common example is the famous Sudoku games, though it has a stronger restriction that each block can have no repeating symbols as well. Relating it back to the 36 officer problem, we introduce the concept of Mutually Orthogonal Latin Squares, where there are a collection of Latin Squares of order $n$ that when superimposed, do not have any repetition of symbols in 2 cells. We explain with an example.

\[
\begin{aligned}
    &L^{(1)} = \begin{bmatrix}
        1 & 2 & 3 & 4\\
        2 & 1 & 4 & 3\\
        3 & 4 & 1 & 2\\
        4 & 3 & 2 & 1
    \end{bmatrix}\quad 
    &L^{(2)} = \begin{bmatrix}
        1 & 2 & 3 & 4\\
        4 & 3 & 2 & 1\\
        2 & 1 & 4 & 3\\
        3 & 4 & 1 & 2
    \end{bmatrix}\quad
    &L^{(3)} = \begin{bmatrix}
        1&2&3&4\\
        3&4&1&2\\
        4&3&2&1\\
        2&1&4&3
    \end{bmatrix}
\end{aligned}
\label{MOLS of order 4}
\]

Here we have a set of Mutually Orthogonal Latin Squares of order 4, or MOLS(4). We can indeed verify that all 3 matrices are Latin Squares as no row or column has a repeating element. We now show orthogonality for the first 2 matrices by this rule:
\begin{align*}
    L^{\{1,2\}} &= [(a_{ij}, b_{ij})]\\
    &= \begin{bmatrix}
        (1,1) & (2,2) & (3,3) & (4,4)\\
        (2,4) & (1,3) & (4,2) & (3,1)\\
        (3,2) & (4,1) & (1,4) & (2,3)\\
        (4,3) & (3,4) & (2,1) & (1,2)
    \end{bmatrix}
\end{align*}
where $L^{(1)}=[a_{ij}]$ and $L^{(2)} = [b_{ij}]$ are the first pair of orthogonal latin squares shown above. The same procedure can be repeated to show orthogonality between $L^{(1)},L^{(3)}$ and $L^{(2)},L^{(3)}$. 

\[
\begin{aligned}
    L^{\{1,3\}} =
    \begin{bmatrix}
        (1,1) & (2,2) & (3,3) & (4,4)\\
        (2,3) & (1,4) & (4,1) & (3,2)\\
        (3,4) & (4,3) & (1,2) & (2,1)\\
        (4,2) & (3,1) & (2,4) & (1,3)
    \end{bmatrix}\quad\quad
    L^{\{2,3\}} =
    \begin{bmatrix}
        (1,1) & (2,2) & (3,3) & (4,4)\\
        (4,3) & (3,4) & (2,1) & (1,2)\\
        (2,4) & (1,3) & (4,2) & (3,1)\\
        (3,2) & (4,1) & (1,4) & (2,3)
    \end{bmatrix}
\end{aligned}
\]As we can see, this is clearly the problem that Euler deemed impossible, just of an order of 6. 

\newpage
Interestingly, it has been proven that for any prime power $q=p^k$, there exist $q-1$ MOLS$(q)$. The construction uses the finite field $\mathbb{F}_q$, where each Latin square is defined using elements $a_i,a_j,a_k\in\mathbb{F}_q$. For the $k$-th Latin square $L^{(k)}$ of order $q$, the entry in row $i$ and column $j$ is given by
\begin{equation*}
    L^{(k)}_{i,j} = a_i+a_ka_j,\quad\text{for }i,j\in[q], \text{ }k\in[q-1].
\end{equation*}

For any Latin Square $L=[a_{ij}]$, we write it as an array of the following form:
\[
\begin{aligned}
    \begin{bmatrix}
        1&1&\dots&1&2&2&\dots&3&\dots&n&n&\dots&n\\
        1&2&\dots&n&1&2&\dots&1&\dots&1&2&\dots&n\\
        a_{11}&a_{12}&\dots&a_{1n}&a_{21}&a_{22}&\dots&a_{31}&\dots&a_{n1}&a_{n2}&\dots&a_{nn}
    \end{bmatrix} \in \mathbb{N}^{3\times n^2}
\end{aligned}
\]
where the first and second row correspond to the $(i,j)$-th position of the element $a_{ij}$, which is positioned on the third row. This is an example of a orthogonal array of size $(3,n)$. Orthogonal arrays can actually be generalised to sizes $(m,n)$, and we study the properties by using incidence structures.

\subsection{From Orthogonal Arrays to Graphs}
For any orthogonal array OA$(m,n)$, we can construct block graphs using the columns of OA$(m.n)$, where 2 columns are adjacent if the columns have overlapping entries in any row. Interestingly enough, this construction of the orthogonal array block graph results in a Strongly Regular Graph \cite{Asgarli_2022}. In this paper we focus on the base case OA$(2,n)$.

We first display OA$(2,n)$:
\[
\begin{aligned}
    \text{OA}(2,n)\begin{bmatrix}
        1&1&\dots&1&2&2&\dots&2&3&\dots&n&n&\dots&n\\
        1&2&\dots&n&1&2&\dots&n&1&\dots&1&2&\dots&n\\
    \end{bmatrix} \in \mathbb{N}^{2\times n^2}
\end{aligned}
\]
We can interpret this combinatorially as sets of ordered pairs $\{1,2,\dots,n\}\times\{1,2,\dots,n\}$, each appearing once in each column of OA$(2,n)$. Following the block graph construction of this graph, any 2 columns are adjacent if the top row of the 2 columns are the same or the bottom row of the 2 columns are the same. This block graph construction of OA$(2,n)$ is precisely the Rook's Graph $R(n)$, which is the graph of how a rook moves on a $n\times n$ chessboard. We can see that each column can represent the $(i,j)$-th position of the chessboard, each columns joined by where the rook can move next.

\subsection{Strongly Regular Graphs}
Along with Rook Graphs, we also consider the Triangular Graph $T(n)$, which is also a strongly regular graph as our family of graphs in this paper. A \textbf{strongly regular graph} with parameters $(v,k,\lambda,\mu)$ is a simple, undirected graph on $v$ vertices such that each vertex has exactly $k$ neighbors, every pair of adjacent vertices shares $\lambda$ common neighbors, and every pair of non-adjacent vertices shares $\mu$ common neighbors.~\cite{godsil2001algebraic}. Strongly regular graphs have many interesting structures and properties, one important property being its relation between its strongly regular parameters and its adjacency matrix $A$, namely $A^2 = kI + \lambda A + \mu(J-I-A)$. A famous example of a strongly regular graph is the Petersen Graph, $\operatorname{SRG(10,3,0,1)}$. A visual representation of the Petersen graph can be found in Figure~\ref{fig:petersen}. The regularity of strongly regular graphs leads to adjacency matrices that generate a commutative algebra of dimension three, which induces a coherent closure~\cite{bannai1984algebraic, greaves2024coherentrankgrapheigenvalues}. 

% \subsection{Primary Goal}
% In their paper, Greaves and Yip \cite{greaves2024coherentrankgrapheigenvalues} studied graphs with 3 eigenvalues and the change in coherent rank upon switching blocks of the respective graphs. There they discovered that switching graphs of 3 eigenvalues resulted in large coherent rank, with some ranks being unbounded. As such, we choose to investigate if it is possible to obtain a general coherent closure when switching strongly regular graphs, as strongly regular graphs are a subset of graphs with 3 eigenvalues. We choose the rook graph and triangular graph, well-known examples of strongly regular graphs, and destroy their symmetry to obtain a more general coherent configuration. In particular, we use Seidel switching and vertex deletion to investigate the coherent rank of graphs modified from strongly regular graphs.

\subsection{Primary Goal}

It is well known that any regular graph with exactly three eigenvalues has a minimum coherent algebra of dimension three, and hence coherent rank 3. However, the behavior of \emph{nonregular} graphs with three eigenvalues is not well understood in this context. In particular, there is no known bound on how large the coherent rank of such graphs can be. One recent construction showing unbounded coherent rank involves switching cliques in block graphs derived from orthogonal arrays~\cite{greaves2024coherentrankgrapheigenvalues}.

This motivates the broader study of how small perturbations to graphs with low coherent rank — especially those with high symmetry — affect their coherent closures. In this project, we focus specifically on the case of the rook graph $R(n)$, which is a strongly regular graph with three eigenvalues and coherent rank 3. The rook graph also arises as the block graph of the orthogonal array OA$(2,n)$, which is the simplest possible orthogonal array construction. 

We investigate whether applying graph operations, such as Seidel switching and vertex deletion, to $R(n)$ and to the triangular graph $T(n)$ can result in a general coherent configuration with significantly higher coherent rank. Our work explores whether these structured yet minimal modifications are sufficient to break the algebraic symmetry in a way that increases the complexity of the coherent closure. In doing so, we aim to contribute to the broader question: \emph{How does switching graphs, initially with low coherent rank, change its coherent rank?}
