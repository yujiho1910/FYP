\documentclass{article}
\usepackage{amsmath, amssymb, amsthm}
\usepackage{geometry}
\usepackage{graphicx}
\usepackage{tikz}
\geometry{a4paper, margin=1in}
\usepackage{titlesec}
\usepackage{hyperref}
\setcounter{MaxMatrixCols}{20}
\newtheorem{proposition}{Proposition}
\newtheorem{definition}{Definition}
\newtheorem{theorem}{Theorem}
\newtheorem{example}{Example}
\newtheorem{property}{Property}

\title{Showing Coherent Algebra}
\author{Ho Jing Rui}
\date{\today}

\begin{document}

\maketitle
\newpage

\section*{Definition of a Rook Graph}

A rook graph is defined as a graph, \(R_{m,n} = (V_R,E_R), \quad m\leq n\), where:
\begin{itemize}
    \item Vertices represent the cells of an \( m \times n \) chessboard, \(|V_R| = mn\).
    \item Two vertices are adjacent if they lie in the same row or column on the chessboard.
    \item The resulting adjacency matrix is as follows:
    \begin{align*}
        \mathbf{A}(R_{m,n}) = \begin{bmatrix}
            J-I & I & \cdots & I \\
            I & J-I & \cdots & I \\
            \vdots & \vdots & \ddots & \vdots \\
            I & I & \cdots & J-I
        \end{bmatrix}
    \end{align*}
    where each block is of size $n\times n$, and there are $m$ blocks on each row and column.
\end{itemize}

\section*{Resulting Matrix Decomposition}

Similarly to how we switch 1 vertex from $R_{n,n}$, we switch $k$-blocks of $n$ vertices ($k<\lfloor n/2\rfloor$ by symmetry) from $R_{n,n}$ and have decomposed the matrix into the form:

\begin{align*}
    \mathbf{A}(\Gamma) = \begin{bmatrix}
        \mathbf{A}(R_{k,n}) & C \\
        C^T & \mathbf{A}(R_{n-k, n})
    \end{bmatrix}
\end{align*}

where $C = J_{k\times n-k} \otimes I$ and $\otimes$ is the Kronecker product.

\subsection*{}

We want to show that this resulting graph has rank 12. We do this in 2 steps.

\begin{enumerate}
    \item Show the upper bound by constructing a coherent algebra.
    \item Show the lower bound using the Wielandt principle.
\end{enumerate}

\section{Constructing a Coherent Algebra}

We wish to show that the set

% \begin{align*}
%     \mathcal{W}(\Gamma) = \langle
%     &\begin{bmatrix}
%         I_{kn\times kn} & 0 \\
%         0 & 0
%     \end{bmatrix},
%     \begin{bmatrix}
%         I_{k\times k} \otimes (J-I)_{n\times n} & 0 \\
%         0 & 0
%     \end{bmatrix},
%     \begin{bmatrix}
%         (J-I)_{k\times k} \otimes I_{n\times n} & 0 \\
%         0 & 0
%     \end{bmatrix},
%     \begin{bmatrix}
%         (J-I)_{k\times k} \otimes (J-I)_{n\times n} & 0 \\
%         0 & 0
%     \end{bmatrix}, \\
%     &\begin{bmatrix}
%         0 & 0 \\
%         0 & I_{n(n-k)\times n(n-k)}
%     \end{bmatrix},
%     \begin{bmatrix}
%         0 & 0 \\
%         0 & I_{n-k\times n-k} \otimes (J-I)_{n\times n}
%     \end{bmatrix},
%     \begin{bmatrix}
%         0 & 0 \\
%         0 & (J-I)_{n-k\times n-k} \otimes I_{n\times n}
%     \end{bmatrix},
%     \begin{bmatrix}
%         0 & 0 \\
%         0 & (J-I)_{n-k\times n-k} \otimes (J-I)_{n\times n}
%     \end{bmatrix}, \\
%     &\begin{bmatrix}
%         0 & C \\
%         0 & 0
%     \end{bmatrix},
%     \begin{bmatrix}
%         0 & J-C \\
%         0 & 0
%     \end{bmatrix}, \\
%     &\begin{bmatrix}
%         0 & 0 \\
%         C^T & 0
%     \end{bmatrix},
%     \begin{bmatrix}
%         0 & 0 \\
%         J-C^T & 0
%     \end{bmatrix},
%     \rangle
% \end{align*}

\begin{align*}
    \mathcal{W}(\Gamma) = \langle
    &\begin{bmatrix}
        I & 0 \\
        0 & 0
    \end{bmatrix},
    \begin{bmatrix}
        I \otimes (J-I) & 0 \\
        0 & 0
    \end{bmatrix},
    \begin{bmatrix}
        (J-I) \otimes I & 0 \\
        0 & 0
    \end{bmatrix},
    \begin{bmatrix}
        (J-I) \otimes (J-I) & 0 \\
        0 & 0
    \end{bmatrix}, \\
    &\begin{bmatrix}
        0 & 0 \\
        0 & I
    \end{bmatrix},
    \begin{bmatrix}
        0 & 0 \\
        0 & I \otimes (J-I)
    \end{bmatrix},
    \begin{bmatrix}
        0 & 0 \\
        0 & (J-I) \otimes I
    \end{bmatrix},
    \begin{bmatrix}
        0 & 0 \\
        0 & (J-I) \otimes (J-I)
    \end{bmatrix}, \\
    &\begin{bmatrix}
        0 & J\otimes I \\
        0 & 0
    \end{bmatrix},
    \begin{bmatrix}
        0 & J\otimes (J-I) \\
        0 & 0
    \end{bmatrix}, 
    \begin{bmatrix}
        0 & 0 \\
        J\otimes I & 0
    \end{bmatrix},
    \begin{bmatrix}
        0 & 0 \\
        J\otimes (J-I) & 0
    \end{bmatrix}
    \rangle
\end{align*}

satisfies the following axioms:

\subsection*{Axioms of a Coherent Configuration}

\textbf{Coherent Configuration Axioms:}

\begin{itemize}
    \item[(CC1)] \quad \( \sum_{i=1}^{r} A_i = J \).
    \item[(CC2)] \quad For each \( i \in \{1, \dots, r\} \), there exists \( j \in \{1, \dots, r\} \) such that \( A_i^T = A_j \).
    \item[(CC3)] \quad There exists a subset \( \Delta \subset \{1, \dots, r\} \) such that \( \sum_{i \in \Delta} A_i = I \).
    \item[(CC4)] \quad \( A_i A_j = \sum_{k=1}^{r} p^k_{i,j} A_k \), for each \( i, j \in \{1, \dots, r\} \).
\end{itemize}

\textbf{Coherent Algebra Axioms:}

A \textit{coherent algebra} is a matrix algebra \( \mathcal{A} \subseteq \text{Mat}_X(\mathbb{C}) \) that satisfies the following axioms:

\begin{itemize}
    \item[(A1)] \quad \( I, J \in \mathcal{A} \).
    \item[(A2)] \quad \( M^T \in \mathcal{A} \) for each \( M \in \mathcal{A} \).
    \item[(A3)] \quad \( MN \in \mathcal{A} \) and \( M \circ N \in \mathcal{A} \) for each \( M, N \in \mathcal{A} \), where \( \circ \) denotes the entrywise product.
\end{itemize}

We do so by splitting up the set $\mathcal{W}(\Gamma)$ into subsets for easier reference:

\begin{align*}
    \mathcal{W}(A_1) = &\langle \begin{bmatrix}
        I & 0 \\
        0 & 0
    \end{bmatrix},
    \begin{bmatrix}
        I \otimes (J-I) & 0 \\
        0 & 0
    \end{bmatrix},
    \begin{bmatrix}
        (J-I) \otimes I & 0 \\
        0 & 0
    \end{bmatrix},
    \begin{bmatrix}
        (J-I) \otimes (J-I) & 0 \\
        0 & 0
    \end{bmatrix}
    \rangle, \\ 
    =&\langle M_{11}, M_{12}, M_{13}, M_{14} \rangle \\ \\
    \mathcal{W}(A_2) = &\langle \begin{bmatrix}
        0 & 0 \\
        0 & I
    \end{bmatrix},
    \begin{bmatrix}
        0 & 0 \\
        0 & I \otimes (J-I)
    \end{bmatrix},
    \begin{bmatrix}
        0 & 0 \\
        0 & (J-I) \otimes I
    \end{bmatrix},
    \begin{bmatrix}
        0 & 0 \\
        0 & (J-I) \otimes (J-I)
    \end{bmatrix}
    \rangle, \\
    =&\langle M_{21}, M_{22}, M_{23}, M_{24}\rangle \\ \\
    \mathcal{W}(C) = &\langle \begin{bmatrix}
        0 & J\otimes I \\
        0 & 0
    \end{bmatrix},
    \begin{bmatrix}
        0 & J\otimes (J-I) \\
        0 & 0
    \end{bmatrix}
    \rangle, \\
    =&\langle M_{31}, M_{32} \rangle \\ \\
    \mathcal{W}(C^T) = &\langle \begin{bmatrix}
        0 & 0 \\
        J\otimes I & 0
    \end{bmatrix},
    \begin{bmatrix}
        0 & 0 \\
        J\otimes (J-I) & 0
    \end{bmatrix}
    \rangle \\
    =&\langle M_{41}, M_{42}\rangle
\end{align*}

\begin{itemize}
    \item We can observe that the elements in $\mathcal{W}(\Gamma)$ are partitions of the matrix $J$, and as such (CC1) is fulfilled.
    \item Each element in $\mathcal{W}(A_1)$ and $\mathcal{W}(A_2)$ is self-transpose, and each element in $\mathcal{W}(C)$ has its transpose in $\mathcal{W}(C^T)$ (i.e. $M_{3i} = M_{4i}^T$, $i \in\{1,2\}$), so (CC2) is also fulfilled, which also fulfils (A2).
    \item $M_{11} + M_{21} = I$, which shows the existence of a subset of $\mathcal{W}(\Gamma)$ such that the sum of the elements in the subset $\Delta = \{M_{11}, M_{21}\}$ is the identity matrix, $I$. This fulfils (CC3) and since (CC1) is also fulfilled, we have shown $I,J \in \mathcal{W}(\Gamma)$, so (A1) is fulfilled. 
    \item We just need to show (CC4) and (A3), but showing (A3) also shows (CC4), so we will choose to show that instead of both.
\end{itemize}

% M◦N ∈ A
\subsection{Entrywise product} 
We start with the trivial part, the entrywise product. We observe that each element in $\mathcal{W}(\Gamma)$ are partitions of $J$, with no overlapping entries that are both non-zero. As such, for any matrix $M,N \in \mathcal{W}(\Gamma)$

\begin{align*}
    M\circ N = \mathbf{0}, \quad\text{ where } \mathbf{0} \text{ is the all-zeros matrix}
\end{align*}

Thus, we have shown for any matrix $M,N \in \mathcal{W}(\Gamma)$, $M\circ N \in \mathcal{W}(\Gamma)$

% MN ∈ A
\subsection{Matrix Multiplication}

We will be using the following proposition to "ignore" the commutative multiplications:

\begin{proposition}\label{prop:commutative_kronecker}
    Given matrices $A,B,C,D$ of compatible dimensions such that $AC=CA \text{ and } BD=DB$, it follows that $(A \otimes B)(C \otimes D) = (C\otimes D)(A\otimes B)$.
\end{proposition}

\begin{proof} Starting from that assumption that $AC=CA \text{ and } BD=DB$,
\begin{align*}
    (A \otimes B)(C \otimes D) &= AC \otimes BD \\
    &= CA\otimes DB \\
    &= (C\otimes D)(A\otimes B)
\end{align*}
\end{proof}


We will first consider the multiplications of the elements within their subsets, $\mathcal{W}(A_1), \mathcal{W}(A_2), \mathcal{W}(C), \mathcal{W}(C^T)$. To simplify the calculations, we will be using the respective block matrices to represent the actual matrices, as given by:

\begin{align*}
    \mathcal{W'}(A_1) 
    &= \langle I_{kn\times kn}, I_{k\times k} \otimes (J-I),(J-I)_{k\times k} \otimes I,(J-I)_{k\times k} \otimes (J-I) \rangle \\
    &= \langle M'_{11}, M'_{12}, M'_{13}, M'_{14} \rangle \\ \\
    \mathcal{W'}(A_2) 
    &=\langle I_{n(n-k)\times n(n-k)}, I_{n-k\times n-k} \otimes (J-I),(J-I)_{n-k\times n-k} \otimes I,(J-I)_{n-k\times n-k} \otimes (J-I)\rangle, \\ 
    &=\langle M'_{21}, M'_{22}, M'_{23}, M'_{24} \rangle \\ \\
    \mathcal{W'}(C) 
    &= \langle J_{k\times n-k}\otimes I, J_{k\times n-k}\otimes (J-I)\rangle \\
    &= \langle M'_{31}, M'_{32} \rangle \\ \\
    \mathcal{W'}(C^T) 
    &= \langle J_{n-k \times k}\otimes I, J_{n-k \times k}\otimes (J-I)\rangle \\
    &= \langle M'_{41}, M'_{42} \rangle
\end{align*}

\begin{itemize}
    \item $\mathcal{W}(A_1)$ \\
    Since the block matrices in $\mathcal{W}(A_1)$ are all in the same position, we can isolate the non-zero block of the matrices, i.e. $\begin{bmatrix}
        A & 0 \\ 0 & 0
    \end{bmatrix}\begin{bmatrix}
        B&0\\0&0
    \end{bmatrix} = \begin{bmatrix}
        AB&0\\0&0
    \end{bmatrix}\neq \mathbf{0}$, so we use the set $\mathcal{W'}(A_1)$.
    
    We know that $M'_{11}M'_{1i} = M'_{1i}M'_{11} = M'_{1i}$ since $M'_{11} = I$, so $M_{11}M_{1i} = M_{1i}M_{11} = M_{1i} \in \mathcal{W}(\Gamma)$.
    
    We now consider the multiplications between $M'_{12},M'_{13}\text{ and }M'_{14}$.
    \begin{itemize}
        \item $M_{12}M_{13}$ \\
        \begin{align*}
            M'_{12}M'_{13} &= (I \otimes (J-I))((J-I) \otimes I) \\
            &= (I(J-I))\otimes((J-I)(I)) \\
            &= (J-I)\otimes(J-I) \\
            &= M'_{14} \\\\
            \Rightarrow M_{12}M_{13} &= M_{14} \in \mathcal{W}(\Gamma)
        \end{align*}
        
        \item $M_{13}M_{12}$ \\ \\
        Let $A = I, C = J-I, B = J-I, D=I$, we can see that $AC=CA \text{ and } BD=DB$. Using Proposition \ref{prop:commutative_kronecker}, 
        \begin{align*}
            M'_{13}M'_{12} &= M'_{12}M'_{13} \\
            &= M'_{14} \\\\
            \Rightarrow M_{13}M_{12} &= M_{14}\in\mathcal{W}(\Gamma)
        \end{align*} \\
        
        \item $M_{12}M_{14}$ 
        \begin{align*}
            M'_{12}M'_{14} 
            &= (I \otimes (J-I))((J-I) \otimes (J-I)) \\
            &= (I(J-I))\otimes((J-I)(J-I)) \\
            &= (J-I)\otimes((n-2)J + I) \\
            &= (n-2)((J-I)\otimes J) + (J-I)\otimes I \\
            &= (n-2)((J-I)\otimes(J-I)) + (n-2)((J-I)\otimes I) + (J-I)\otimes I \\
            &= (n-2)M'_{14} + (n-1)M'_{13} \\\\
            \Rightarrow M_{12}M_{14} &= (n-2)M_{14} + (n-1)M_{13} \in \mathcal{W}(\Gamma)
        \end{align*}
        
        \item $M_{14}M_{12}$ \\ \\
        Let $A = I, C = J-I, B = J-I, D=J-I$, we can see that $AC=CA \text{ and } BD=DB$. Using Proposition \ref{prop:commutative_kronecker}, 
        \begin{align*}
            M'_{14}M'_{12} &= M'_{12}M'_{14} \\
            &= (n-2)M'_{14} + (n-1)M'_{13} \\ \\
            \Rightarrow M_{14}M_{12} &= (n-2)M_{14} + (n-1)M_{13} \in \mathcal{W}(\Gamma)
        \end{align*} \\
        
        \item $M_{13}M_{14}$ 
        \begin{align*}
            M'_{13}M'_{14} 
            &=((J-I) \otimes I)((J-I) \otimes (J-I)) \\
            &= ((J-I)(J-I))\otimes(I(J-I)) \\
            &= ((n-2)J + I)\otimes (J-I) \\
            &= (n-2)(J\otimes (J-I)) + I\otimes (J-I) \\
            &= (n-2)((J-I)\otimes(J-I)) + (n-2)(I\otimes (J-I)) + I\otimes (J-I) \\
            &= (n-2)M'_{14} + (n-1)M'_{12}  \\ \\
            \Rightarrow M_{13}M_{14} &= (n-2)M_{14} + (n-1)M_{12} \in \mathcal{W}(\Gamma)
        \end{align*}

        \item $M_{14}M_{13}$ \\ \\
        Let $A = J-I, C = J-I, B = I, D=J-I$, we can see that $AC=CA \text{ and } BD=DB$. Using Proposition \ref{prop:commutative_kronecker}, 
        \begin{align*}
            M'_{14}M'_{13} 
            &= M'_{13}M'_{14} \\
            &= (n-2)M'_{14} + (n-1)M'_{12} \\ \\
            \Rightarrow M_{14}M_{13} &= (n-2)M_{14} + (n-1)M_{12} \in \mathcal{W}(\Gamma)
        \end{align*} \\
    \end{itemize}
    So we have shown that the matrices $M,N \in \mathcal{W}(A_1)$ satisfy the property $MN \in \mathcal{W}(A_1) \subset \mathcal{W}(\Gamma)$.

    \item $\mathcal{W}(A_2)$ \\
    Since the block matrices in $\mathcal{W}(A_2)$ are all in the same position, we can isolate the non-zero block of the matrices, i.e. $\begin{bmatrix}
        0 & 0 \\ 0 & A
    \end{bmatrix}\begin{bmatrix}
        0&0\\0&B
    \end{bmatrix} = \begin{bmatrix}
        0&0\\0&AB
    \end{bmatrix}\neq \mathbf{0}$, so we use the set $\mathcal{W'}(A_2)$.

    Note that $\mathcal{W'}(A_2) = \mathcal{W'}(A_1)$ and by following the working above, we can derive that the matrices  $M,N \in \mathcal{W}(A_2)$ satisfy the property that $MN \in \mathcal{W}(A_2) \subset \mathcal{W}(\Gamma)$

    \item $\mathcal{W}(C)$ \\
    Since the block matrices in $\mathcal{W}(C)$ are all in the same position, we can isolate the non-zero block of the matrices, i.e. $\begin{bmatrix}
        0 & A \\ 0 & 0
    \end{bmatrix}\begin{bmatrix}
        0&B\\0&0
    \end{bmatrix} = \begin{bmatrix}
        0&0\\0&0
    \end{bmatrix}$. This shows that no matter which matrices $M,N \in \mathcal{W}(C)$ we choose, $MN=\mathbf{0} \in \mathcal{W}(\Gamma)$.

    \item $\mathcal{W}(C^T)$ \\
    Similar to $\mathcal{W}(C)$, we show that $\begin{bmatrix}
        0 & 0 \\ A & 0
    \end{bmatrix}\begin{bmatrix}
        0&0\\B&0
    \end{bmatrix} = \begin{bmatrix}
        0&0\\0&0
    \end{bmatrix}$, showing that no matter which matrices $M,N \in \mathcal{W}(C^T)$ we choose, $MN=\mathbf{0} \in \mathcal{W}(\Gamma)$.
\end{itemize}

We now consider multiplications between different subset partitions of $\mathcal{W}(\Gamma)$.

\begin{itemize}
    \item $\mathcal{W}(A_1)$ and $\mathcal{W}(A_2)$ \\
    \begin{itemize}
        \item For $M\in\mathcal{W}(A_1),N \in \mathcal{W}(A_2)$, matrix multiplications would be of the form \\
        \begin{align*}
            MN = \begin{bmatrix}
                A & 0 \\ 0 & 0
            \end{bmatrix}\begin{bmatrix}
                0&0\\0&B
            \end{bmatrix} = \begin{bmatrix}
                0&0\\0&0
            \end{bmatrix}
        \end{align*}
        This shows that for any matrices $M\in\mathcal{W}(A_1),N \in \mathcal{W}(A_2)$, the product would be $\mathbf{0}\in\mathcal{W}(\Gamma)$.

        \item For $M\in\mathcal{W}(A_2),N \in \mathcal{W}(A_1)$, matrix multiplications would be of the form \\
        \begin{align*}
            MN = \begin{bmatrix}
                0 & 0 \\ 0 & A
            \end{bmatrix}\begin{bmatrix}
                B&0\\0&0
            \end{bmatrix} = \begin{bmatrix}
                0&0\\0&0
            \end{bmatrix}
        \end{align*}
        This shows that for any matrices $M\in\mathcal{W}(A_2),N \in \mathcal{W}(A_1)$, the product would also be $\mathbf{0}\in\mathcal{W}(\Gamma)$.
    \end{itemize}
    Thus, for any 2 matrices $M,N$ from subsets $\mathcal{W}(A_1)$ and $\mathcal{W}(A_2)$, the product $MN \in \mathcal{W}(\Gamma)$.

    \item $\mathcal{W}(A_1)$ and $\mathcal{W}(C)$ \\
    \begin{itemize}
        \item For $M\in\mathcal{W}(A_1),N \in \mathcal{W}(C)$, matrix multiplications would be of the form \\
        \begin{align*}
            MN = \begin{bmatrix}
                A & 0 \\ 0 & 0
            \end{bmatrix}\begin{bmatrix}
                0&B\\0&0
            \end{bmatrix} = \begin{bmatrix}
                0&AB\\0&0
            \end{bmatrix} \neq \mathbf{0}
        \end{align*}
        so we use the sets $\mathcal{W'}(A_1)$ and $\mathcal{W'}(C)$.
        
        We know that $M'_{11}M'_{3i} = M'_{3i}$ since $M'_{11} = I$, so $M_{11}M_{31} =M_{31}\text{ and } M_{11}M_{32} = M_{32}$. So both $M_{11}M_{31}, M_{11}M_{32}  \in \mathcal{W}(C) \subset\mathcal{W}(\Gamma)$.
        \begin{itemize}
            \item $M'_{12}M'_{31}$ \\
            \begin{align*}
                M'_{12}M'_{31}
                &= (I\otimes(J-I)(J\otimes I) \\
                &= IJ\otimes(J-I)I \\
                &= J\otimes (J-I) = M'_{32} \\ \\
                \Rightarrow M_{12}M_{31} &= M_{32} \in \mathcal{W}(C) \subset \mathcal{W}(\Gamma)
            \end{align*}
            
            \item $M'_{12}M'_{32}$ \\
            \begin{align*}
                M'_{12}M'_{32}
                &= (I\otimes(J-I)(J\otimes (J-I)) \\
                &= IJ\otimes(J-I)(J-I) \\
                &= J\otimes((n-2)J + I)\\
                &= (n-2)J\otimes J + J\otimes I\\
                &= (n-2)J\otimes(J-I) +(n-2)J\otimes I + J\otimes I\\
                &= (n-2)J\otimes(J-I) +(n-1)J\otimes I \\
                &= (n-2)M'_{32} +(n-1)M'_{31}\\ \\
                \Rightarrow M_{12}M_{32} &= (n-2)M_{32} +(n-1)M_{31} \in \mathcal{W}(C) \subset \mathcal{W}(\Gamma)
            \end{align*}

            \item $M_{13}M_{31}$ \\
            \begin{align*}
                M'_{13}M'_{31}
                &= ((J-I)\otimes I)(J\otimes I) \\
                &= (J-I)_{k\times k}J_{k\times n-k}\otimes II \\
                &= (k-1)J\otimes I\\
                &= (k-1)M'_{31} \\ \\
                \Rightarrow M_{13}M_{31} &= (k-1)M_{31} \in \mathcal{W}(C) \subset \mathcal{W}(\Gamma)
            \end{align*}

            \item $M_{13}M_{32}$ \\
            \begin{align*}
                M'_{13}M'_{32}
                &= ((J-I)\otimes I)(J\otimes (J-I)) \\
                &= (J-I)_{k\times k}J_{k\times n-k}\otimes I(J-I) \\
                &= (k-1)J\otimes (J-1)\\
                &= (k-1)M'_{32} \\ \\
                \Rightarrow M_{13}M_{32} &= (k-1)M_{32} \in \mathcal{W}(C) \subset \mathcal{W}(\Gamma)
            \end{align*}

            \item $M_{14}M_{31}$ \\
            \begin{align*}
                M'_{14}M'_{31}
                &= ((J-I)\otimes (J-I))(J\otimes I) \\
                &= (J-I)_{k\times k}J_{k\times n-k}\otimes (J-I)I \\
                &= (k-1)J\otimes (J-I)\\
                &= (k-1)M'_{32} \\ \\
                \Rightarrow M_{14}M_{31} &= (k-1)M_{32} \in \mathcal{W}(C) \subset \mathcal{W}(\Gamma)
            \end{align*}

            \item $M_{14}M_{32}$ \\
            \begin{align*}
                M'_{14}M'_{32}
                &= ((J-I)\otimes (J-I))(J\otimes (J-I)) \\
                &= (J-I)_{k\times k}J_{k\times n-k}\otimes (J-I)(J-I) \\
                &= (k-1)J\otimes ((n-2)J+I)\\
                &= (k-1)(n-2)J\otimes J + (k-1)J\otimes I \\
                &= (k-1)(n-2)J\otimes (J-I) + (k-1)(n-2)J\otimes I + (k-1)J\otimes I\\
                &= (k-1)(n-2)J\otimes (J-I) + (k-1)(n-1)J\otimes I \\
                &= (k-1)(n-2)M'_{32} + (k-1)(n-1)M'_{31} \\ \\
                \Rightarrow M_{14}M_{32} &= (k-1)(n-2)M_{32} + (k-1)(n-1)M_{31} \in \mathcal{W}(C) \subset \mathcal{W}(\Gamma)
            \end{align*}
        \end{itemize}
        
        \item For $M\in\mathcal{W}(C),N \in \mathcal{W}(A_1)$, matrix multiplications would be of the form \\
        \begin{align*}
            MN = \begin{bmatrix}
                0 & A \\ 0 & 0
            \end{bmatrix}\begin{bmatrix}
                B&0\\0&0
            \end{bmatrix} = \begin{bmatrix}
                0&0\\0&0
            \end{bmatrix}
        \end{align*}
        This shows that for any matrices $M\in\mathcal{W}(C),N \in \mathcal{W}(A_1)$, the product would also be $\mathbf{0}\in\mathcal{W}(\Gamma)$.
    \end{itemize}
    Thus, for any 2 matrices $M,N$ from subsets $\mathcal{W}(A_1)$ and $\mathcal{W}(C)$, the product $MN \in \mathcal{W}(\Gamma)$.

    \item $\mathcal{W}(A_1)$ and $\mathcal{W}(C^T)$ \\
    \begin{itemize}
        \item For $M\in\mathcal{W}(A_1),N \in \mathcal{W}(C^T)$, matrix multiplications would be of the form \\
        \begin{align*}
            MN = \begin{bmatrix}
                A & 0 \\ 0 & 0
            \end{bmatrix}\begin{bmatrix}
                0&0\\B&0
            \end{bmatrix} = \begin{bmatrix}
                0&0\\0&0
            \end{bmatrix}
        \end{align*}
        This shows that for any matrices $M\in\mathcal{W}(A_1),N \in \mathcal{W}(C^T)$, the product would also be $\mathbf{0}\in\mathcal{W}(\Gamma)$.

        \item For $M\in\mathcal{W}(C^T),N \in \mathcal{W}(A_1)$, matrix multiplications would be of the form \\
        \begin{align*}
            MN = \begin{bmatrix}
                0 & 0 \\ A & 0
            \end{bmatrix}\begin{bmatrix}
                B&0\\0&0
            \end{bmatrix} = \begin{bmatrix}
                0&0\\AB&0
            \end{bmatrix}\neq \mathbf{0}
        \end{align*}
        so we use the sets $\mathcal{W'}(C^T)$ and $\mathcal{W'}(A_1)$. \\
        
        We know that $M'_{4i}M'_{11} = M'_{4i}$ since $M'_{11} = I$, so $M_{41}M_{11} = M_{41} \text{ and } M_{42}M_{11} = M_{42}$. So both $M_{41}M_{11}, M_{42}M_{11} \in \mathcal{W}(C^T) \subset\mathcal{W}(\Gamma)$.

        \begin{itemize}
            \item $M_{41}M_{12}$ \\
            \begin{align*}
                M'_{41}M'_{12}
                &= (J_{n-k\times k}\otimes I)(I_{k\times k}\otimes(J-I)) \\
                &= (J_{n-k\times k}I_{k\times k})\otimes(I(J-I)) \\
                &= J\otimes (J-I) \\
                &= M'_{42} \\ \\
                \Rightarrow M_{41}M_{12} &= M_{42} \in \mathcal{W}(C^T) \subset \mathcal{W}(\Gamma)
            \end{align*}

            \item $M_{42}M_{12}$ \\
            \begin{align*}
                M'_{42}M'_{12}
                &= (J_{n-k\times k}\otimes (J-I))(I_{k\times k}\otimes(J-I)) \\
                &= (J_{n-k\times k}I_{k\times k})\otimes((J-I)(J-I)) \\
                &= J\otimes ((n-2)J + I) \\
                &= (n-2)J\otimes J +  J\otimes I\\
                &= (n-2)J\otimes(J-I) + (n-2)J\otimes I + J\otimes I\\
                &= (n-2)M'_{42} + (n-1)M'_{41} \\ \\
                \Rightarrow M_{42}M_{12} &= (n-2)M_{42} + (n-1)M_{41} \in \mathcal{W}(C^T) \subset \mathcal{W}(\Gamma)
            \end{align*}

            \item $M_{41}M_{13}$ \\
            \begin{align*}
                M'_{41}M'_{13}
                &= (J_{n-k\times k}\otimes I)((J-I)_{n\times n}\otimes I) \\
                &= (J_{n-k\times k}(J-1)_{k\times k})\otimes I(I) \\
                &= (k-1)J\otimes I\\
                &= (k-1)M'_{41} \\ \\
                \Rightarrow M_{41}M_{13} &= (k-1)M_{41} \in \mathcal{W}(C^T) \subset \mathcal{W}(\Gamma)
            \end{align*}
            
            \item $M_{42}M_{13}$ \\
            \begin{align*}
                M'_{42}M'_{13}
                &= (J_{n-k\times k}\otimes (J-I))((J-I)_{n\times n}\otimes I) \\
                &= (J_{n-k\times k}(J-1)_{k\times k})\otimes (J-I)(I) \\
                &= (k-1)J\otimes (J-I)\\
                &= (k-1)M'_{42} \\ \\
                \Rightarrow M_{42}M_{13} &= (k-1)M_{42} \in \mathcal{W}(C^T) \subset \mathcal{W}(\Gamma)
            \end{align*}

            \item $M_{41}M_{14}$ \\
            \begin{align*}
                M'_{41}M'_{14}
                &= (J_{n-k\times k}\otimes I)((J-I)_{k\times k}\otimes (J-I)) \\
                &= (J_{n-k\times k}(J-1)_{k\times k})\otimes I(J-I) \\
                &= (k-1)J\otimes (J-I)\\
                &= (k-1)M'_{42} \\ \\
                \Rightarrow M_{41}M_{14} &= (k-1)M_{42} \in \mathcal{W}(C^T) \subset \mathcal{W}(\Gamma)
            \end{align*}

            \item $M_{42}M_{14}$ \\
            \begin{align*}
                M'_{42}M'_{14}
                &= (J_{n-k\times k}\otimes (J-I))((J-I)_{k\times k}\otimes (J-I)) \\
                &= (J_{n-k\times k}(J-I)_{k\times k})\otimes((J-I)(J-I)) \\
                &= (k-1)J\otimes((n-2)J + I) \\
                &= (k-1)(n-2)J\otimes J + (k-1)J\otimes I \\
                &= (k-1)(n-2)J\otimes (J-I) +(k-1)(n-2)J\otimes I + (k-1)J\otimes I \\
                &= (k-1)(n-2)J\otimes(J-I) +(k-1)(n-1)J\otimes I \\
                &= (k-1)(n-2)M'_{42} + (k-1)(n-1)M'_{41} \\ \\
                \Rightarrow M_{42}M_{14} &= (k-1)(n-2)M_{42} + (k-1)(n-1)M_{41} \in \mathcal{W}(C^T) \subset \mathcal{W}(\Gamma)
            \end{align*}
        \end{itemize}
        This shows that for any matrices $M\in\mathcal{W}(C^T),N \in \mathcal{W}(A_1)$, the product $MN\in\mathcal{W}(\Gamma)$.
    \end{itemize}
    Thus, for any 2 matrices $M,N$ from subsets $\mathcal{W}(A_1)$ and $\mathcal{W}(C^T)$, the product $MN \in \mathcal{W}(\Gamma)$.

    \item $\mathcal{W}(A_2)$ and $\mathcal{W}(C)$ \\
    \begin{itemize}
        \item For $M\in\mathcal{W}(A_2),N \in \mathcal{W}(C)$, matrix multiplications would be of the form \\
        \begin{align*}
            MN = \begin{bmatrix}
                0 & 0 \\ 0 & A
            \end{bmatrix}\begin{bmatrix}
                0&B\\0&0
            \end{bmatrix} = \begin{bmatrix}
                0&0\\0&0
            \end{bmatrix}
        \end{align*}
        This shows that for any matrices $M\in\mathcal{W}(A_2),N \in \mathcal{W}(C)$, the product $MN = \mathbf{0}\in\mathcal{W}(\Gamma)$.

        \item For $M\in\mathcal{W}(C),N \in \mathcal{W}(A_2)$, matrix multiplications would be of the form \\
        \begin{align*}
            MN = \begin{bmatrix}
                0 & A \\ 0 & 0
            \end{bmatrix}\begin{bmatrix}
                0&0\\0&B
            \end{bmatrix} = \begin{bmatrix}
                0&AB\\0&0
            \end{bmatrix} \neq \mathbf{0}
        \end{align*}
        so we use the sets $\mathcal{W'}(C)$ and $\mathcal{W'}(A_2)$. \\
        We know that $M'_{3i}M'_{21} = M'_{3i}$ since $M'_{21} = I$, so $M_{31}M_{21} = M_{31} \text{ and } M_{32}M_{21} = M_{32}$. So both $M_{31}M_{21}, M_{32}M_{21} \in \mathcal{W}(C) \subset\mathcal{W}(\Gamma)$.
        \begin{itemize}
            \item $M_{31}M_{22}$
            \begin{align*}
                (J_{k\times n-k}\otimes I)(I_{n-k\times n-k}\otimes (J-I))
                &= J_{k\times n-k}I_{n-k\times n-k} \otimes I(J-I) \\
                &= J_{k\times n-k}\otimes (J-I) \\
                &= M'_{32} \\ \\
                \Rightarrow M_{31}M_{12} &= M_{32} \in \mathcal{W}(C) \subset \mathcal{W}(\Gamma)
            \end{align*}
            \item $M_{32}M_{22}$
            \begin{align*}
                M'_{32}M'_{22}
                &= (J_{k\times n-k}\otimes (J-I))(I_{n-k\times n-k} \otimes (J-I)) \\
                &= J_{k\times n-k}I_{k\times n-k} \otimes (J-I)(J-I) \\
                &= J_{k\times n-k}\otimes((n-2)J + I) \\
                &= (n-2)J_{k\times n-k}\otimes J + J_{k\times n-k}\otimes I\\
                &= (n-2)J_{k\times n-k}\otimes (J-I) + (n-2)J_{k\times n-k}\otimes I + J_{k\times n-k}\otimes I\\
                &= (n-2)M'_{32} + (n-1)M'_{31} \\\\
                \Rightarrow M_{32}M_{22} &= (n-2)M_{32} + (n-1)M_{31} \in \mathcal{W}(C) \subset \mathcal{W}(\Gamma)
            \end{align*}
            \item $M_{31}M_{23}$
            \begin{align*}
                M'_{31}M'_{23}
                &= (J_{k\times n-k}\otimes I)((J-I)_{n-k\times n-k} \otimes I) \\
                &= J_{k\times n-k}(J-I)_{n-k\times n-k} \otimes I(I) \\
                &= (n-k-1)J_{k\times n-k}\otimes I \\
                &= (n-k-1)M'_{31} \\\\
                \Rightarrow M_{31}M_{23} &= (n-k-1)M_{31} \in \mathcal{W}(C) \subset\mathcal{W}(\Gamma)
            \end{align*}
            \item $M_{32}M_{23}$
            \begin{align*}
                M'_{32}M'_{23}
                &= (J_{k\times n-k}\otimes (J-I))((J-I)_{n-k\times n-k} \otimes I) \\
                &= J_{k\times n-k}(J-I)_{n-k\times n-k} \otimes (J-I)I \\
                &= (n-k-1)J_{k\times n-k}\otimes (J-I) \\
                &= (n-k-1)M'_{32} \\\\
                \Rightarrow M_{32}M_{23} &= (n-k-1)M_{32} \in \mathcal{W}(C) \subset\mathcal{W}(\Gamma)
            \end{align*}
            \item $M_{31}M_{24}$
            \begin{align*}
                M'_{31}M'_{24}
                &= (J_{k\times n-k}\otimes I)((J-I)_{n-k\times n-k} \otimes (J-I)) \\
                &= J_{k\times n-k}(J-I)_{n-k\times n-k} \otimes I(J-I) \\
                &= (n-k-1)J_{k\times n-k}\otimes (J-I) \\
                &= (n-k-1)M'_{32} \\\\
                \Rightarrow M_{31}M_{24} &= (n-k-1)M_{32} \in \mathcal{W}(C) \subset\mathcal{W}
            \end{align*}
            \item $M_{32}M_{24}$
            \begin{align*}
                M'_{32}M'_{24}
                &= (J_{k\times n-k}\otimes (J-I))((J-I)_{n-k\times n-k} \otimes (J-I)) \\
                &= J_{k\times n-k}(J-I)_{n-k\times n-k} \otimes (J-I)(J-I) \\
                &= (n-k-1)J_{k\times n-k}\otimes ((n-2)J + I) \\
                &= (n-k-1)(n-2)J_{k\times n-k}\otimes J + (n-k-1)J_{k\times n-k}\otimes I\\
                &= (n-k-1)(n-2)J_{k\times n-k}\otimes (J-I) + (n-k-1)(n-2)J_{k\times n-k}\otimes I \\
                &\quad+ (n-k-1)J_{k\times n-k}\otimes I\\
                &= (n-k-1)(n-2)M'_{32} + (n-k-1)(n-1)M'_{31} \\\\
                \Rightarrow M_{32}M_{24} &= (n-k-1)(n-2)M'_{32} + (n-k-1)(n-1)M'_{31} \in \mathcal{W}(C) \subset\mathcal{W}
            \end{align*}
        \end{itemize}
        This shows that for any matrices $M\in\mathcal{W}(C),N \in \mathcal{W}(A_2)$, the product $MN\in\mathcal{W}(\Gamma)$.
    \end{itemize}
    Thus, for any 2 matrices $M,N$ from subsets $\mathcal{W}(A_2)$ and $\mathcal{W}(C)$, the product $MN \in \mathcal{W}(\Gamma)$.

    \item $\mathcal{W}(A_2)$ and $\mathcal{W}(C^T)$ \\
    \begin{itemize}
        \item For $M\in\mathcal{W}(A_2),N \in \mathcal{W}(C^T)$, matrix multiplications would be of the form \\
        \begin{align*}
            MN = \begin{bmatrix}
                0 & 0 \\ 0 & A
            \end{bmatrix}\begin{bmatrix}
                0&0\\B&0
            \end{bmatrix} = \begin{bmatrix}
                0&0\\AB&0
            \end{bmatrix} \neq \mathbf{0}
        \end{align*}
        so we use the sets $\mathcal{W'}(A_2)$ and $\mathcal{W'}(C^T)$. \\
        \begin{itemize}
            \item $M_{22}M_{41}$
            \begin{align*}
                M'_{22}M'_{41}
                &= (I_{n-k\times n-k} \otimes (J-I))(J_{n-k\times k}\otimes I) \\
                &= I_{n-k\times n-k}J_{n-k\times k} \otimes (J-I)I \\
                &= J_{n-k\times k} \otimes (J-I) \\
                &= M'_{42} \\\\
                \Rightarrow M_{22}M_{41} &= M_{42} \in \mathcal{W}(C^T) \subset\mathcal{W}(\Gamma)
            \end{align*}
            
            \item $M_{22}M_{42}$
            \begin{align*}
                M'_{22}M'_{42}
                &= (I_{n-k\times n-k} \otimes (J-I))(J_{n-k\times k}\otimes (J-I)) \\
                &= I_{n-k\times n-k}J_{n-k\times k} \otimes (J-I)(J-I) \\
                &= J_{n-k\times k} \otimes ((n-2)J+I) \\
                &= (n-2)J_{n-k\times k}\otimes J + J\otimes I \\
                &= (n-2)J_{n-k\times k}\otimes(J-I) + (n-2)J\otimes I + J\otimes I \\
                &= (n-2)J_{n-k\times k}\otimes (J-I) + (n-1)J\otimes I \\
                &= (n-2)M'_{42} + (n-1)M'_{41} \\\\
                \Rightarrow M_{22}M_{42} &= (n-2)M_{42} + (n-1)M_{41} \in \mathcal{W}(C^T)\subset\mathcal{W}(\Gamma)
            \end{align*}
            
            \item $M_{23}M_{41}$
            \begin{align*}
                M'_{23}M'_{41}
                &= ((J-I)_{n-k\times n-k} \otimes I)(J_{n-k\times k}\otimes I) \\
                &= (J-I)_{n-k\times n-k}J_{n-k\times k}\otimes I(I) \\
                &= (n-k-1)J_{n-k\times k}\otimes I\\
                &= (n-k-1)M'_{41} \\\\
                \Rightarrow M_{23}M_{41} &= (n-k-1)M_{41}\in\mathcal{W}(C^T)\subset\mathcal{W}(\Gamma)
            \end{align*}
            
            \item $M_{23}M_{42}$
            \begin{align*}
                M'_{23}M'_{42}
                &= ((J-I)_{n-k\times n-k} \otimes I)(J_{n-k\times k}\otimes (J-I)) \\
                &= (J-I)_{n-k\times n-k}J_{n-k\times k}\otimes I(J-I) \\
                &= (n-k-1)J_{n-k\times k}\otimes (J-I)\\
                &= (n-k-1)M'_{42} \\\\
                \Rightarrow M_{23}M_{42} &= (n-k-1)M_{42}\in\mathcal{W}(C^T)\subset\mathcal{W}(\Gamma)
            \end{align*}
            
            \item $M_{23}M_{41}$
            \begin{align*}
                M'_{24}M'_{41}
                &= ((J-I)_{n-k\times n-k} \otimes (J-I))(J_{n-k\times k}\otimes I) \\
                &= (J-I)_{n-k\times n-k}J_{n-k\times k}\otimes (J-I)I \\
                &= (n-k-1)J_{n-k\times k}\otimes (J-I)\\
                &= (n-k-1)M'_{42} \\\\
                \Rightarrow M_{24}M_{41} &= (n-k-1)M_{42}\in\mathcal{W}(C^T)\subset\mathcal{W}(\Gamma)
            \end{align*}
            
            \item $M_{23}M_{42}$
            \begin{align*}
                M'_{24}M'_{42}
                &= ((J-I)_{n-k\times n-k} \otimes (J-I))(J_{n-k\times k}\otimes (J-I)) \\
                &= (J-I)_{n-k\times n-k}J_{n-k\times k}\otimes (J-I)(J-I) \\
                &= (n-k-1)J_{n-k\times k}\otimes ((n-2)J+I)\\
                &= (n-k-1)(n-2)J_{n-k\times k}\otimes J + (n-k-1)J_{n-k\times k}\otimes I \\
                &= (n-k-1)(n-2)J_{n-k\times k}\otimes (J-I) + (n-k-1)(n-2)J\otimes I \\
                &\quad +(n-k-1)J\otimes I \\
                &= (n-k-1)(n-2)M'_{42} + (n-k-1)(n-1)M'_{41} \\\\
                \Rightarrow M_{24}M_{42} &= (n-k-1)(n-2)M_{42} + (n-k-1)(n-1)M_{41} \in\mathcal{W}(C^T)\subset\mathcal{W}(\Gamma)
            \end{align*}
        \end{itemize}
        This shows that for any matrices $M\in\mathcal{W}(A_2),N \in \mathcal{W}(C^T)$, the product $MN \in\mathcal{W}(\Gamma)$.
        
        \item For $M\in\mathcal{W}(C^T),N \in \mathcal{W}(A_2)$, matrix multiplications would be of the form \\
        \begin{align*}
            MN = \begin{bmatrix}
                0 & 0 \\ A & 0
            \end{bmatrix}\begin{bmatrix}
                0&0\\0&B
            \end{bmatrix} = \begin{bmatrix}
                0&0\\0&0
            \end{bmatrix}
        \end{align*}
        This shows that for any matrices $M\in\mathcal{W}(C^T),N \in \mathcal{W}(A_2)$, the product $MN = \mathbf{0}\in\mathcal{W}(\Gamma)$.
    \end{itemize}
    Thus, for any 2 matrices $M,N$ from subsets $\mathcal{W}(A_2)$ and $\mathcal{W}(C^T)$, the product $MN \in \mathcal{W}(\Gamma)$.

    \item $\mathcal{W}(C)$ and $\mathcal{W}(C^T)$ \\
    \begin{itemize}
        \item For $M\in\mathcal{W}(C),N \in \mathcal{W}(C^T)$, matrix multiplications would be of the form \\
        \begin{align*}
            MN = \begin{bmatrix}
                0 & A \\ 0 & 0
            \end{bmatrix}\begin{bmatrix}
                0&0\\B&0
            \end{bmatrix} = \begin{bmatrix}
                AB&0\\0&0
            \end{bmatrix} \neq \mathbf{0}
        \end{align*}
        so we use the sets $\mathcal{W'}(C)$ and $\mathcal{W'}(C^T)$. \\
        \begin{itemize}
            \item $M_{31}M_{41}$
            \begin{align*}
                M'_{31}M'_{41}
                &= (J_{k\times(n-k)}\otimes I)(J_{(n-k)\times k}\otimes I) \\
                &= J_{k\times(n-k)}J_{(n-k)\times k} \otimes I(I) \\
                &= (n-k)J_{k\times k}\otimes I \\
                &= (n-k)(J-I)_{k\times k}\otimes I + (n-k)I_{k\times k}\otimes I \\
                &= (n-k)(J-I)_{k\times k}\otimes I + (n-k)I_{kn\times kn} \\
                &= (n-k)M'_{13} + (n-k)M'_{11} \\\\
                \Rightarrow M_{31}M_{41} &= (n-k)M_{13} + (n-k)M_{11} \in \mathcal{W}(A_1)\subset\mathcal{W}(\Gamma)
            \end{align*}
            
            \item $M_{31}M_{42}$
            \begin{align*}
                M'_{31}M'_{42}
                &= (J_{k\times(n-k)}\otimes I)(J_{(n-k)\times k}\otimes (J-I)) \\
                &= J_{k\times(n-k)}J_{(n-k)\times k} \otimes I(J-I) \\
                &= (n-k)J_{k\times k}\otimes (J-I) \\
                &= (n-k)(J-I)_{k\times k}\otimes (J-I) + (n-k)I_{k\times k}\otimes (J-I) \\
                &= (n-k)M'_{14} + (n-k)M'_{12} \\\\
                \Rightarrow M_{31}M_{42} &= (n-k)M_{14} + (n-k)M_{12} \in \mathcal{W}(A_1)\subset\mathcal{W}(\Gamma)
            \end{align*}
            
            \item $M_{32}M_{41}$
            \begin{align*}
                M'_{32}M'_{41}
                &= (J_{k\times(n-k)}\otimes (J-I))(J_{(n-k)\times k}\otimes I) \\
                &= J_{k\times(n-k)}J_{(n-k)\times k} \otimes (J-I)I \\
                &= (n-k)J_{k\times k}\otimes (J-I) \\
                &= (n-k)(J-I)_{k\times k}\otimes (J-I) + (n-k)I_{k\times k}\otimes (J-I) \\
                &= (n-k)M'_{14} + (n-k)M'_{12} \\\\
                \Rightarrow M_{32}M_{41} &= (n-k)M_{14} + (n-k)M_{12} \in \mathcal{W}(A_1)\subset\mathcal{W}(\Gamma)
            \end{align*}
            
            \item $M_{32}M_{42}$
            \begin{align*}
                M'_{32}M'_{42}
                &= (J_{k\times(n-k)}\otimes (J-I))(J_{(n-k)\times k}\otimes (J-I)) \\
                &= J_{k\times(n-k)}J_{(n-k)\times k} \otimes (J-I)(J-I) \\
                &= (n-k)J_{k\times k}\otimes ((n-2)J+I) \\
                &= (n-k)(n-2)J_{k\times k}\otimes J + (n-k)J_{k\times k}\otimes I
            \end{align*}
            Note that $J_{k\times k}\otimes I = M'_{13} + M'_{11}$ from $M_{31}M_{41}$, so we break down $J_{k\times k}\otimes J$:
            \begin{align*}
                J_{k\times k}\otimes J
                &= J_{k\times k}\otimes (J-I) + J_{k\times k}\otimes I \\
                &= (J-I)_{k\times k}\otimes (J-I) + I_{k\times k}\otimes (J-I) + M'_{13} + M'_{11} \\
                &= M'_{14}+M'_{12}+M'_{13} + M'_{11}
            \end{align*}
            Combining,
            \begin{align*}
                M'_{32}M'_{42}
                &=(n-k)(n-2)J_{k\times k}\otimes J + (n-k)J_{k\times k}\otimes I \\
                &= (n-k)(n-2)(M'_{11}+M'_{12}+M'_{13} + M'_{14}) + (n-k)(M'_{11} + M'_{13}) \\\\
                \Rightarrow M_{32}M_{42} &= (n-k)(n-2)(M_{11}+M_{12}+M_{13} + M_{14}) + (n-k)(M_{11} + M_{13}) \\
                \Rightarrow M_{32}M_{42} &\in\mathcal{W}(A_1)\subset\mathcal{W}(\Gamma)
            \end{align*}
        \end{itemize}
        This shows that for any matrices $M\in\mathcal{W}(C),N \in \mathcal{W}(C^T)$, the product $MN \in\mathcal{W}(\Gamma)$.

        \item For $M\in\mathcal{W}(C^T),N \in \mathcal{W}(C)$, matrix multiplications would be of the form \\
        \begin{align*}
            MN = \begin{bmatrix}
                0 & 0 \\ A & 0
            \end{bmatrix}\begin{bmatrix}
                0&B\\0&0
            \end{bmatrix} = \begin{bmatrix}
                0&0\\0&AB
            \end{bmatrix} \neq \mathbf{0}
        \end{align*}
        so we use the sets $\mathcal{W'}(C^T)$ and $\mathcal{W'}(C)$. \\
        \begin{itemize}
            \item $M_{41}M_{31}$
            \begin{align*}
                M'_{41}M'_{31}
                &= (J_{(n-k)\times k}\otimes I)(J_{k\times (n-k)}\otimes I) \\
                &= J_{(n-k)\times k}J_{k\times (n-k)} \otimes I(I) \\
                &= kJ_{(n-k)\times (n-k)}\otimes I \\
                &= k((J-I)_{(n-k)\times (n-k)}\otimes I + I_{(n-k)\times (n-k)}\otimes I )\\
                &= k((J-I)_{(n-k)\times (n-k)}\otimes I + I_{(n-k)n\times (n-k)n} )\\
                &= k(M'_{23} + M'_{21}) \\\\
                \Rightarrow M_{41}M_{31} &= k(M_{23} + M_{21}) \in \mathcal{W}(A_2)\subset\mathcal{W}(\Gamma)
            \end{align*}
            
            \item $M_{41}M_{32}$
            \begin{align*}
                M'_{41}M'_{32}
                &= (J_{(n-k)\times k}\otimes I)(J_{k\times (n-k)}\otimes (J-I)) \\
                &= J_{(n-k)\times k}J_{k\times (n-k)} \otimes I(J-I) \\
                &= kJ_{(n-k)\times (n-k)}\otimes (J-I) \\
                &= k((J-I)_{(n-k)\times (n-k)}\otimes (J-I) + I_{(n-k)\times (n-k)}\otimes (J-I) )\\
                &= k(M'_{24} + M'_{22}) \\\\
                \Rightarrow M_{41}M_{32} &= k(M_{24} + M_{22}) \in \mathcal{W}(A_2)\subset\mathcal{W}(\Gamma)
            \end{align*}
            
            \item $M_{42}M_{31}$
            \begin{align*}
                M'_{42}M'_{31}
                &= (J_{(n-k)\times k}\otimes (J-I))(J_{k\times (n-k)}\otimes I) \\
                &= J_{(n-k)\times k}J_{k\times (n-k)} \otimes (J-I)I \\
                &= kJ_{(n-k)\times (n-k)}\otimes (J-I) \\
                &= k((J-I)_{(n-k)\times (n-k)}\otimes (J-I) + I_{(n-k)\times (n-k)}\otimes (J-I) )\\
                &= k(M'_{24} + M'_{22}) \\\\
                \Rightarrow M_{42}M_{31} &= k(M_{24} + M_{22}) \in \mathcal{W}(A_2)\subset\mathcal{W}(\Gamma)
            \end{align*}
            
            \item $M_{42}M_{32}$
            \begin{align*}
                M'_{42}M'_{32}
                &= (J_{(n-k)\times k}\otimes (J-I))(J_{k\times (n-k)}\otimes (J-I)) \\
                &= J_{(n-k)\times k}J_{k\times (n-k)} \otimes (J-I)(J-I) \\
                &= kJ_{(n-k)\times (n-k)}\otimes ((n-2)J+I) \\
                &= k(n-2)J_{(n-k)\times (n-k)}\otimes J + kJ_{(n-k)\times (n-k)}\otimes I
            \end{align*}
            Note that $kJ_{(n-k)\times (n-k)}\otimes I = k(M'_{23} + M'_{21})$ from $M_{41}M_{31}$, so we break down $J_{(n-k)\times (n-k)}\otimes J$:
            \begin{align*}
                J_{(n-k)\times (n-k)}\otimes J
                &= J_{(n-k)\times (n-k)}\otimes (J-I) + J_{(n-k)\times (n-k)}\otimes I \\
                &= (J-I)_{(n-k)\times (n-k)}\otimes (J-I) + I_{(n-k)\times (n-k)}\otimes (J-I) + M'_{23} + M'_{21} \\
                &= M'_{24}+M'_{22}+M'_{23} + M'_{21}
            \end{align*}
            Combining,
            \begin{align*}
                M'_{42}M'_{32}
                &= k(n-2)J_{(n-k)\times (n-k)}\otimes J + kJ_{(n-k)\times (n-k)}\otimes I \\
                &= k(n-2)(M'_{21}+M'_{22}+M'_{23} + M'_{24}) + k(M'_{21} + M'_{23}) \\\\
                \Rightarrow M_{42}M_{32} &= k(n-2)(M_{21}+M_{22}+M_{23} + M_{24}) + k(M_{21} + M_{23}) \\
                \Rightarrow M_{42}M_{32} &\in\mathcal{W}(A_2)\subset\mathcal{W}(\Gamma)
            \end{align*}
        \end{itemize}
        This shows that for any matrices $M\in\mathcal{W}(C^T),N \in \mathcal{W}(C)$, the product $MN \in\mathcal{W}(\Gamma)$.
    \end{itemize}
    Thus, for any 2 matrices $M,N$ from subsets $\mathcal{W}(C)$ and $\mathcal{W}(C^T)$, the product $MN \in \mathcal{W}(\Gamma)$.
\end{itemize}
Finally, we have shown that (A3) is fulfilled, and have shown that $\mathcal{W}(\Gamma)$ is a coherent configuration and a coherent algebra.

As a result, the coherent rank of this graph $r$ has an upper bound of 12.

\section{Showing Lower Bound}
Recall the coherent algebra given by:

% \begin{align*}
%     \mathcal{W}(\Gamma) = \langle
%     &\begin{bmatrix}
%         I_{kn\times kn} & 0 \\
%         0 & 0
%     \end{bmatrix},
%     \begin{bmatrix}
%         I_{k\times k} \otimes (J-I) & 0 \\
%         0 & 0
%     \end{bmatrix},
%     \begin{bmatrix}
%         (J-I)_{k\times k} \otimes I & 0 \\
%         0 & 0
%     \end{bmatrix},
%     \begin{bmatrix}
%         (J-I)_{k\times k} \otimes (J-I) & 0 \\
%         0 & 0
%     \end{bmatrix}, \\
%     &\begin{bmatrix}
%         0 & 0 \\
%         0 & I_{(n-k)n\times (n-k)n}
%     \end{bmatrix},
%     \begin{bmatrix}
%         0 & 0 \\
%         0 & I_{(n-k)\times (n-k)} \otimes (J-I)
%     \end{bmatrix},
%     \begin{bmatrix}
%         0 & 0 \\
%         0 & (J-I)_{(n-k)\times (n-k)} \otimes I
%     \end{bmatrix},
%     \begin{bmatrix}
%         0 & 0 \\
%         0 & (J-I)_{(n-k)\times (n-k)} \otimes (J-I)
%     \end{bmatrix}, \\
%     &\begin{bmatrix}
%         0 & J_{k\times (n-k)}\otimes I \\
%         0 & 0
%     \end{bmatrix},
%     \begin{bmatrix}
%         0 & J_{k\times (n-k)}\otimes (J-I) \\
%         0 & 0
%     \end{bmatrix}, 
%     \begin{bmatrix}
%         0 & 0 \\
%         J_{(n-k)\times k}\otimes I & 0
%     \end{bmatrix},
%     \begin{bmatrix}
%         0 & 0 \\
%         J_{(n-k)\times k}\otimes (J-I) & 0
%     \end{bmatrix}
%     \rangle
% \end{align*}
\begin{align*}
    \mathcal{W}(\Gamma) = \langle
    &\begin{bmatrix}
        I & 0 \\
        0 & 0
    \end{bmatrix},
    \begin{bmatrix}
        I \otimes (J-I) & 0 \\
        0 & 0
    \end{bmatrix},
    \begin{bmatrix}
        (J-I) \otimes I & 0 \\
        0 & 0
    \end{bmatrix},
    \begin{bmatrix}
        (J-I) \otimes (J-I) & 0 \\
        0 & 0
    \end{bmatrix}, \\
    &\begin{bmatrix}
        0 & 0 \\
        0 & I
    \end{bmatrix},
    \begin{bmatrix}
        0 & 0 \\
        0 & I \otimes (J-I)
    \end{bmatrix},
    \begin{bmatrix}
        0 & 0 \\
        0 & (J-I) \otimes I
    \end{bmatrix},
    \begin{bmatrix}
        0 & 0 \\
        0 & (J-I) \otimes (J-I)
    \end{bmatrix}, \\
    &\begin{bmatrix}
        0 & J\otimes I \\
        0 & 0
    \end{bmatrix},
    \begin{bmatrix}
        0 & J\otimes (J-I) \\
        0 & 0
    \end{bmatrix}, 
    \begin{bmatrix}
        0 & 0 \\
        J\otimes I & 0
    \end{bmatrix},
    \begin{bmatrix}
        0 & 0 \\
        J\otimes (J-I) & 0
    \end{bmatrix}
    \rangle
\end{align*}
Note that for every two matrices $M,N\in\mathcal{W}(\Gamma)$, its entrywise product is $\mathbf{0}$, i.e.
\begin{align*}
    M\circ N = \mathbf{0}, \quad M,N\in\mathcal{W}(\Gamma)
\end{align*}

Since the elements in $\mathcal{W}(\Gamma)$ are 0s and 1s, it shows that the elements have no overlapping entries that are 1.

As such, if we remove any matrix from this set, the resulting matrices will not sum to $J$, which fails to satisfy the axiom (CC1)
If we remove $A_k\in \mathcal{W}(\Gamma)$, resulting in a smaller set we call $\mathcal{W}(\Gamma')$, we will have
\begin{align*}
    \Sigma_{i=1}^{r'} A_i= J -A_k,\quad |\mathcal{W}(\Gamma')| = r'
\end{align*}

Therefore, the set $\mathcal{W}(\Gamma)$ and its elements form a minimal coherent algebra, and thus the lower bound of its rank $r$ is 12.

\section*{Conclusion}
Since we have shown the upper and loewr bound of the coherent rank of the graph $\Gamma$ to be 12 in both cases, we conclude that the coherent rank of $\Gamma$ is 12, i.e. $cr(\Gamma) = |\mathcal{W}(\Gamma)| = 12$
\end{document}